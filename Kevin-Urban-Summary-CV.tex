% vim:tw=72 sw=2 ft=tex
%         File: text.tex
% Date Created: 2015 Jul 27
%  Last Change: 2017 Mar 20
%     Compiler: pdflatex
%       Author: Kurban

\documentclass[10pt]{article}

%%%%%%%%%%%%%%%%%%%%%%%%%%%%%%%%%%%%%%%%%%%%
%%              Packages                  %%
%%%%%%%%%%%%%%%%%%%%%%%%%%%%%%%%%%%%%%%%%%%%
\usepackage[top=0.5in,bottom=0.5in,left=0.5in,right=0.5in]{geometry}
\usepackage[usenames,dvipsnames]{color}
%\usepackage[colorlinks=true,urlcolor=blue]{hyperref}
\usepackage{hyperref}
%% For making compact lists.
% -  use itemize* environment for compact itemized lists, etc.
\usepackage{mdwlist}

%%%%%%%%%%%%%%%%%%%%%%%%%%%%%%%%%%%%%%%%%%%%
%%               New Commands             %%
%%%%%%%%%%%%%%%%%%%%%%%%%%%%%%%%%%%%%%%%%%%%
%% For making a new section in your resume. %%
% -  \noindent prevents LaTeX from indenting.
% -  \large makes the text slightly larger.
%    See http://en.wikibooks.org/wiki/LaTeX/Fonts#Sizing_text
%    for a list of other font-sizing commands.
% -  \vspace adds a little bit of vertical space.
% -  \hrule adds a horizontal line.
\newcommand{\ressection}[1]{\noindent{\large\textbf{#1}}
\vspace{2pt}\hrule\vspace{4pt}}
%% For job titles and date names. %%
% -  \hfill fills up horizontal space until
%    the words to the left and the right are
%    as far apart as possible.
\newcommand{\leftandright}[2]{\noindent\textbf{#1}\hfill
\textbf{#2}}

%%%%%%%%%%%%%%%%%%%%%%%%%%%%%%%%%%%%%%%%%%%%%
%%           Miscellaneous                 %%
%%%%%%%%%%%%%%%%%%%%%%%%%%%%%%%%%%%%%%%%%%%%%
\pagestyle{empty} %% removes the page numbers.



%%%%%%%%%%%%%%%%%%%%%%%%%%%%%%%%%%%%%%%%%%%%%
%%             The Resume!                 %%
%%%%%%%%%%%%%%%%%%%%%%%%%%%%%%%%%%%%%%%%%%%%%
\begin{document}


% (1) Mandeep & Rich @ Antuit felt "Dr. Kevin Urban" was best w/ Education
% at bottom
% (2) Recruiter from Audible was horrified by "Dr" and weirded out that
% Education was at the bottom. He also did not like the profile and
% thought a 2-sentence objective statement would be better.
% (3) a Phd-2-Industry blog I was reading advocated for putting
% Education History and AFTER Work History, which should be sorted by
% relevance, NOT date
%

%\leftandright{\textbf{\huge{Dr. Kevin Urban}}}
\leftandright{\textbf{\huge{Kevin Urban, PhD}}}
%\leftandright{\textbf{\huge{Kevin Urban}}}
{\url{https://www.linkedin.com/in/drkrbn}}

\leftandright{kevin.ddu@gmail.com \textbullet\, (973) 464-6833}
%\leftandright{$\quad$}  % blank for personal website
{\url{https://github.com/krbnite}} 

\leftandright{}
{\url{https://krbnite.github.io/}}


%========================================================
%     OBJECTIVE
%========================================================
% \vspace{0.4cm}
% \begin{center}
%   \textbf{Objective}: To obtain a data scientist role at J.P. Morgan
% \end{center}
% \vspace{-0.5cm}



%========================================================
%     PROFILE / SUMMARY
%========================================================
\vspace{0.5cm}
\ressection{Profile}
% \noindent Innovative data scientist and physicist with proven experience in 
% data product development and integrated, cohesive visualization of complex,
% multi-source, high-dimensional data sets. Intuition for trends and
% patterns which yield discerning insights and hypotheses, while
% strategic insight and methodology result in rapid research-to-product
% development. Self-motivated, collaborative personality with the understanding that 
% a cross-functional, open-minded work environment and
% coupled skill sets result in efficient, high-value, dependable data
% products. Effective written and verbal communicator.
\noindent Highly-motivated, innovative individual with extensive
experience in empirical research, data product
development, and transparent visualization of varied, high-dimensional
data sets. Creative, analytic problem solver equipped with an intuition
for trends and patterns in complex data sets, and insight into the
underlying mathematical structure. Strategic and focused, interested in
rapid research-to- product development. Self-motivated, collaborative
personality with the understanding that a cross-functional, open-minded
work environment and combined skill sets result in efficient delivery of
high-value, dependable data products. Effective written and verbal
communicator.


%========================================================
%     SUMMARY OF QUALIFICATIONS
%========================================================

%========================================================
%     EDUCATION
%========================================================
%\vspace{0.2cm}
%\ressection{Education}
%
%\leftandright{Ph.D., Applied Physics}{May 2016} \\  
%\indent New Jersey Institute of Technology \& Rutgers University, Newark, NJ\\  
%\indent Dissertation: The Hydromagnetic Structure of the Polar Cap and Its Interaction with the Solar Wind
%
%\leftandright{M.S., Applied Physics (Minor: Applied Mathematics)}{August 2010} \\  
%\indent New Jersey Institute of Technology \& Rutgers University, Newark, NJ\\  
%\indent Thesis: Synoptic Variability of a CIR-Driven Open-Closed Boundary During Solar Minimum
%
%
%\leftandright{B.S., Applied Physics (Minor: Applied Mathematics)}{May 2008} \\  
%\indent New Jersey Institute of Technology \& Rutgers University, Newark, NJ\\  



%========================================================
%    DATA / PROGRAMMING LANGUAGES
%========================================================
% -- this is kind of a "Summary of Qualifications"
\vspace{0.5cm}
\ressection{Computing Skills}
\noindent\textbf{Programming / Tools}: Proficient in Python, R, TensorFlow, SQL,
  Amazon Web Services (Redshift, S3, EC2, EBS), Tableau, MatLab, IDL, PostgreSQL, 
  UNIX, Bash, LaTeX, Vim, HTML, CSS, MarkDown; Forays into Fortran, Spark, Hive, Java, 
  C/C++, JavaScript
  \vspace{0.1in}

\noindent\textbf{Analytics}:
\vspace{-0.2cm}
\begin{itemize*}
  \item \textbf{Structured Data} (e.g., sensors, instrumentation, 
    time series, geospatial, simulation, genomics, commerce)
  \item \textbf{Unstructured Data} (e.g., image analysis, logs, text mining,
    web scraping, natural language processing)
  \item \textbf{Data Pre-Processing} (e.g., remedial outlier detection, 
    imputation, aggregation, feature engineering, feature selection)
  \item \textbf{Probability \& Statistics} (e.g., inference,
    hypothesis testing, resampling, conditioning, error analysis)
  \item \textbf{Machine Learning} (e.g., statistical models, neural networks,
    deep learning, regression, classification, clustering, prediction,
    dimensionality reduction, pattern recognition, decision trees)
  \item \textbf{Time Series Analysis} (parametric, nonparametric, spectral
    analysis, signal processing, forecasting)
  \item \textbf{Pattern Recognition} (e.g., kernel density methods,	
    event/anomaly detection)
  \item \textbf{Numerical Modeling}
    (e.g., difference equations, numerical derivatives, numerical
    integration, linear/nonlinear models, dynamical systems, numerical stability, 
    eigenvalue estimation)
\end{itemize*}






%========================================================
%  WORK / RESEARCH   EXPERIENCE
%========================================================
\vspace{0.2cm}
\ressection{Work and Professional Research Experience}

\leftandright{WWE, Advanced Analytics Team}{Stamford, CT \textbullet\, October
2016 -- Present} \\  
\vspace{-0.8em}
\textit{Data Scientist, Manager} 
\begin{itemize*}
  \item Currently leading deep neural network efforts to fully take advantage
    of our content and data assets, developing collaborative and
    item-based recommendation systems, and exploring various customer
    churn models in Python / TensorFlow
  \item Have lead numerous business-driven analytics research and
    predictive modeling projects for WWE Network 
  \item  Used spectral analyses of customer churn/winback behaviors 
    to enhance customer segmentation efforts, help advise email campaigns, and
    improve churn/winback forecasting 
  \item Developed multiple revenue attribution models for WWE Network content
    driven by viewership behaviour at the customer level
  \item Tied fan panel survey response data to customer accounts on the
    WWE Network, allowing us to better quantify the relationship between
    subjective responses and objective viewership data
  \item Conducted sentiment analysis of customer cancellation surveys
    and social media content, uncovering resolvable issues and informing
    how to better serve our customer base
  \item Designed both project-agnostic and project-specific R packages to
    help streamline and/or automate data capture, cleansing,
    transformation, analysis, and reporting. 
  \item Utilized decision tree analysis to help understand and segement
    customers by WWE content and character preferences
  \item Designed interactive dashboards in Tableau 
\end{itemize*}



\leftandright{NJIT Center for Solar-Terrestrial Research}{Newark, NJ \textbullet\, January
2012 -- May 2016} \\  
\vspace{-0.8em}
\textit{Research Scientist} 
\begin{itemize*}
  \item Led multiple data-driven research projects resulting in a
    first-author publication, PhD dissertation, and several papers 
    currently in development
  \item Provided new, empirical insights and interpretation of
    decades-old problem concerning the geomagnetic field's complex
    structure in Earth's polar regions
  \item Developed prediction scheme to forecast the evolution of
    hydromagnetic energy in the deep polar cap
  \item Developed remote-sensing technique using ground-based data sets
    to infer/predict parameters in near-Earth space
  \item Created innovative visualization and analysis techniques
  \item Implemented robust, non-parametric 
    methods to characterize empirical distributions
  \item Conducted the collection, harmonization, and normalization of
    100's of time series data streams from instruments/sensors
    distributed across the Earth and solar system to model and
    understand the linear/nonlinear drivers of a complex systems
  \item Developed software packages to enable efficient, streamlined analysis 
% \item Through the combination of data/observationally-driven research and deep
%   physical understanding of magnetohydrodynamics, explained
%   decades-old problem that prevented proper understanding of the geomagnetic field's
%   complex structure in Earth's polar regions 
%   \textsuperscript{\href{http://www.slideshare.net/inverseuniverse/polarcappower}
%   {[link]}}. The research involved stochastic analysis; development of
%   novel visualization and analysis techniques
%   \textsuperscript{\href{http://www.slideshare.net/inverseuniverse/spectralproducts-64254698}{[link]}};
%   %IDL__8day_multi-mag-spectra.pdf}{example}};
%   collection, harmonization, and normalization of 100's of data sets 
%   \textsuperscript{\href{http://kevin-urban.com/video/R__20130302_trifecta.mp4}{[link]}}
%   distributed across the Earth and solar system to model and
%   understand the linear/nonlinear drivers of complex, physical
%   systems.
% \item The product development included statistical and physical model
%   selection, new interpretations of results commonly understood as
%   anomalies found in the empirical spatial distribution, hypothesis
%   testing, and development of dynamic correlation metrics which verified 
%   time-varying causal relationships 
%   \href{http://kevin-urban.com/images/dynamic-correlograms.pdf}
%   {\textsuperscript{[link]}}, thus verifying the 
%   contrary hypothesis to previous understanding of the subject matter.
\end{itemize*}


%-------------------------------------------------------------
\leftandright{NJIT Department of Mathematical Sciences}{Newark, NJ \textbullet\, 2008 -- 2012}\\  
\vspace{-0.8em}
\textit{Research Scientist, Mathematical Modeling} 
\begin{itemize*}
  \item Extensively modeled, analyzed, and visualized several
    representations of granular fluid systems in MatLab and R
    %\href{http://www.kevin-urban.com/images/gran_tapping-pics.png}
    %{\textsuperscript{\tiny{[link]}}} 
  \item Investigated sources of chaos in a 
    dimensionally-reduced parameter space of a discrete-dynamical system
    (e.g., via \href{http://kevin-urban.com/video/R__gamma-var_restitution-0.8.gif}{Poincare
    map simulations in R \textsuperscript{\tiny{[link]}}})
  \item Led effort in understanding and controlling the propagation of numerical errors, 
    which otherwise prevent proper quantitative analysis and
    visualization of the underlying dynamical system
  \item Co-authored four peer-reviewed papers numerically modeling,
    visualizing, and analyzing granular fluid systems
  \item Led research project on applications of fractional calculus and
    co-authored peer-reviewed publication 
  \item Explored real-world applications of advanced mathematical disciplines such as
    dynamical systems, topological analysis, manifold theory, abstract
    algebra, stochastic calculus
    % \href{http://msp.org/jomms/2011/6-1/jomms-v6-n1-p06-s.pdf}
    % {\textsuperscript{\tiny{[link1],}}}  
    % \href{http://www.sciencedirect.com/science/article/pii/S0167278914000189}
    % {\textsuperscript{\tiny{[link2],}}} 
    % \href{http://www.icmp.lviv.ua/journal/zbirnyk.64/}
    % {\textsuperscript{\tiny{[link3]}}} 
\end{itemize*}

% \vspace{0.25cm}
% \leftandright{Department of Physics and Astronomy, Siena
% College}{Loudonville, NY \textbullet\, August 2011 -- January 2012}\\  
% \vspace{-0.8em}
% \textit{Research Scientist, Contractor} 
% \begin{itemize*}
%   \item Led project continuing the development of automated space weather
%     nowcasting by leveraging time series, spectral, regression, and
%     classification techniques on real-time data from
%     geospatially-distributed network of instrumentation
% \end{itemize*}


%-------------------------------------------------------------
\leftandright{NASA/CalTech Jet Propulsion Laboratory}{Pasadena, CA \textbullet\, Summer 2011}\\  
\vspace{-0.8em}
\textit{Intern, Trajectory Optimization / Design} 
\begin{itemize*}
  \item
    Collaborated with a team of engineers and scientists to develop a full-fledged spacecraft mission to
    the Trojan asteroids of Jupiter
    (from % responding to an Announcement of Opportunity [AO]
    establishing and prioritizing science goals, to optimizing the science-engineering-financial
    parameter space, to the written proposal and presenting our mission design to the NASA review
    board)
  \item 
    Gained an appreciation of rapid product development via concurrent
    engineering, and an understanding of the intricate interplay between the
    various engineering system designs, science goals, timeline
    requirements, and budget constraints.
  \item Published peer-reviewed paper documenting lessons learned
  %\item
  %  Gained understanding of the interplay between the trajectory design
  %  and other considerations, such as launch
  %  windows, budget time lines, a multi-component propulsion system,
  %  and payload
  %\item 
  %  Helped develop the science time line and traceability matrix for the
  %  baseline (desired) and threshold (minimum requirement) missions

\end{itemize*}
   
%-------------------------------------------------------------
\leftandright{Independent Research Consultant}{Clifton, NJ \textbullet\, 
August 2010 - August 2011} \par
\vspace{-0.8em}
\begin{itemize*}
  \item Obtained contracts from NJIT and Siena College to 
    continue the development of automated, real-time
    prediction of geomagnetic parameters
  \item Published peer-reviewed paper documenting proof-of-concept and 
    lessons learned
    \href{http://onlinelibrary.wiley.com/doi/10.1029/2011SW000688/full}
    {proof-of-concept \textsuperscript{\tiny{[link]}}}
    %: Synoptic variability of a CIR-driven
    %open-closed boundary during solar minimum.
\end{itemize*}   


\leftandright{NJIT Department of Physics}{Newark, NJ \textbullet\, 
August 2008 - August 2010} \par
\textit{Research Assistant }
\vspace{-0.8em}
\begin{itemize*}
  \item Led project demonstrating the feasibility of a real-time
    detection/classification scheme in maintaining surveillance and
    prediction of 
    local, regional, and global-scale geomagnetic events of interest
  \item
    Analyzed multi-channel instrument data from a spatially-distributed
    network of automated observatories in Antarctica.
  \item Developed data metrics and associated classification scheme for
    events of interest in the data sets.
  \item In-depth experience visualizing and analyzing data in R, MatLab,
    and IDL
  \item In-depth experience in various data analysis techniques,
    such as digital signal processing, time series analysis, spectral analysis,
    statistical methods, modeling, and regression.
  \item Hardware-software interfacing of instrumentation for data
    collection
  \item Published Master's thesis
\end{itemize*}

%-------------------------------------------------------------
\leftandright{NASA Goddard Space Flight Center }{ }
\leftandright{\textcolor{white}{space} }{Greenbelt, MD \textbullet\, June -- August 2007}\\  
\vspace{-0.8em}
\textit{Intern, Observational Cosmology Laboratory} 
\begin{itemize*}
  \item Developed software implementing a Stokes parameter analysis to
    run on simulation data for the proposed Absolute Spectrum
    Polarimeter [ASP] --- an instrument designed to detect B-mode
    gravitational waves, which would provide evidence for Einstein's
    theory of general relativity and cosmological inflationary theory
  \item Worked with multiple programming languages, including MatLab,
    Bash (shell scripting), and IDL 
  % \item Studied cosmology as it related to the mission goals of ASP
\end{itemize*}


%-------------------------------------------------------------
\leftandright{NASA Goddard Space Flight Center}{Greenbelt, MD \textbullet\, June -- August 2006}\\  
\vspace{-0.8em}
\textit{Intern,  Heliophysics Division}
\begin{itemize*}
  \item Worked extensively in the UNIX programming environment, learning and
    developing Python, HTML, CSS, JavaScript, and PHP code
  \item Developed web content  % for the International Heliophysical Year [IHY]
\end{itemize*}



%========================================================
%     PEER-REVIEWED PAPERS
%========================================================
\vspace{0.2cm}
\ressection{Selected Peer-Reviewed Publications}
\vspace{-0.8em}
\begin{itemize*}

  %\item
  %  % Urban, K. D., A. J. Gerrard, L. J. Lanzerotti, and A. T. Weatherwax, 
  %  Penetration of solar wind hydromagnetic
  %  energy deep into the polar cap: remote detection and forecasting. 
  %  Geophysical Research Letters,
  %  Submitted, 2016.

  \item
    % Urban, K. D., A. J. Gerrard, L. J. Lanzerotti, and A. T. Weatherwax, 
    Rethinking the polar cap: Eccentric-dipole structuring of ULF power
    at the highest corrected geomagnetic latitudes, Journal of Geophysical
    Research, Space Physics, Vol. 121, 2016.

  \item \label{itm:gran1}
    %Blackmore, D., A. Rosato, X. Tricoche, K. D. Urban, and L. Zou,
    %\href{http://www.sciencedirect.com/science/article/pii/S0167278914000189}{}
    Analysis, simulation, and visualization of 1D tapping via
    reduced dynamical models, Physica D: Nonlinear Phenomena, Vol.
    273, 2014.

  \item
    %Diniega, S., K. Sayanagi, J. Balcerski, B. Carande, R. Diaz-Silva,
    %A. Fraeman, S. Guzewich, J. Hudson, A. Nahm, S. Potter, M. Route, K.
    %D. Urban, S. Vasisht, B. Benneke, S. Gil, R. Livi, B. Williams, and C.
    %Budney, L. Lowes,   
    %\href{http://www.sciencedirect.com/science/article/pii/S0032063312003741}{
    Mission to the Trojan Asteroids: Lessons learned
    during a JPL Planetary Science Summer School mission design exercise,
    %\textsuperscript{\tiny{[link]}}}, 
    Planetary and Space Science, Vol. 76, 2013.

  \item 
    %Rosato, A., D. Blackmore, X. M. Tricoche, K. D. Urban, and L. Zuo,
    %\href{http://scitation.aip.org/content/aip/proceeding/aipcp/10.1063/1.4811931}{
    Dynamical systems model and discrete element simulations of a tapped granular
    column,
    %\textsuperscript{\tiny{[link]}}}
    Powders and Grains 2013: Proceedings of
    the 7th International Conference on Micromechanics of Granular
    Media. Vol. 1542. No. 1. AIP Publishing, 2013.

  \item
    %Urban, K. D., Y. Bhattacharya, A. J. Gerrard, A. Ridley, L. Lanzerotti,
    %and A. Weatherwax, 
    %\href{http://onlinelibrary.wiley.com/doi/10.1029/2011SW000688/full}{
    Quiet-time observations of the open-closed boundary
    prior to the CIR-induced storm of August 9, 2008,
    %\textsuperscript{\tiny{[link]}}} 
    Space Weather, Vol. 9, S11001, 2011.

  \item \label{itm:gran2}
    %Blackmore, D., A. Rosato, X. Tricoche, K. Urban, and V. Ratnaswamy,
    %\href{http://msp.berkeley.edu/jomms/2011/6-1/jomms-v6-n1-p06-s.pdf}{
    Tapping dynamics for a column of particles and beyond,
    %\textsuperscript{\tiny{[link]}}}. 
    Journal of Mechanics of Materials and Structures, Vol. 6, No. 1-4, 2011.

  \item \label{itm:gran3}
    % Blackmore, D., K. Urban, and A. Rosato,
    %\href{http://www.icmp.lviv.ua/journal/zbirnyk.64/}{
    Integrability analysis of regular and fractional Blackmore-Samulyak-Rosato fields,
    %\textsuperscript{\tiny{[link]}}}.  
    Condensed Matter Physics, Vol. 13, No. 4, 2010. 
\end{itemize*}




%========================================================
%     EDUCATION
%========================================================
\vspace{0.2cm}
\ressection{Education}

\leftandright{Ph.D., Applied Physics}{$\quad$}%May 2016} \\  
\indent New Jersey Institute of Technology \& Rutgers University, Newark, NJ\\  
\indent Dissertation: The Hydromagnetic Structure of the Polar Cap and Its Interaction with the Solar Wind

\leftandright{M.S., Applied Physics (Minor: Applied Mathematics)}{$\quad$}%{August 2010} \\  
\indent New Jersey Institute of Technology \& Rutgers University, Newark, NJ\\  
\indent Thesis: Synoptic Variability of a CIR-Driven Open-Closed Boundary During Solar Minimum


\leftandright{B.S., Applied Physics (Minor: Applied Mathematics)}{$\quad$}%}{May 2008} \\  
\indent New Jersey Institute of Technology \& Rutgers University, Newark, NJ\\  







%========================================================
%    COMMUNICATION
%========================================================
%\ressection{Miscellaneous}
%\noindent Effective written and verbal communicator, e.g.,
%experiences include:
%\vspace{-0.3cm}
%\begin{itemize*}
%  \item published author
%  \item news writer 
%  \item physics and mathematics tutor
%  \item guitar instructor 
%  \item performing musician
%  \item scored in the 99th percentile in the verbal section of the
%    Graduate Record Examination (GRE)
%\end{itemize*}

%========================================================
%    COURSEWORK
%========================================================
%\ressection{Relevant Coursework}
%\noindent\textbf{Graduate}:
%    Real Analysis, Complex Analysis, Topology, Differentiable Manifolds,
%    Applications of Abstract Algebra, Stochastic Calculus, 
%    Radio Astronomy, Stellar Magnetism, Physics of the Magnetosphere/Ionosphere System,
%    Atmospheric Physics, Electrodynamics, Statistical Mechanics,
%    Classical Mechanics, Quantum Mechanics, Quantum Electrodynamics
%\par\vspace{0.5em}
%\noindent\textbf{Undergraduate}:
%    Data Reduction, Probability/Statistics, Linear Algebra, Computer
%    Science I-II (Java/C++), Abstract Algebra, Vector Calculus,
%    Differential Equations, Advanced Calculus, Electromagnetism,
%    Thermodynamics, Classical Mechanics, Observational Astronomy,
%    Astronomy, Astrophysics I-II,  Quantum Mechanics, Special
%    Relativity, General Relativity
%\par\vspace{0.5em}
%\noindent\textbf{Extracurricular}
%    Computing for Data Analysis (R) [Coursera, R. Peng], Data Analysis
%    (R) [Coursera, J. Leek], Bioinformatics Algorithms [Coursera, P.
%    Pevzner], Machine Learning [Coursera, A. Ng], Operator Algebras and
%    Conformal Field Theory [Summer School, U. Oregon], Concurrent
%    Engineering [Planetary Science Summer School, Jet Propulsion
%    Laboratory]



\end{document}




