
\leftandright{Early Signal (Cohen Veterans Bioscience)}{New York, NY
\textbullet\, July 2018 -- Present} \\  
\leftandright{\emph{Associate Director of Data Science \& Digital Health
(June 2019 - Present)}}{} \\
blasdfasb
\begin{itemize*}
  \item\leftandright
    {\textbf{Maternal Health}}
    {\small{sdfsd}} \\  
    * Report to Gates Foundation
  \item\leftandright
    {\textbf{Sleep Research}}
    {\small{sdfsd}} \\  
    * List posters...
    * UPCOMING ONE...
\end{itemize*}
\leftandright{\emph{Senior Data Scientist (July 2018 - Jun 2019)}}{} \\
blasdfasb
\begin{itemize*}
  \item\leftandright
    {\textbf{Management}}
    * developed data science cookiecutter project template
    * Helped write grant and secure funding for machine learning / suicide
    initiative with Office of Naval Research, Army Analytics Group, and
    University of Virginia
    * helped junior data scientist w/ predictive modeling project by
    introducing him to Box-Cox transformations...
  \item\leftandright
    {\textbf{Gestures}}
    * Research into zero- and one-shot learning, and how they may help the
    gestures project
    * identified sources of data leakage in gestures data splitting
    strategy
    * * gestures data processing pipeline;  fixed leakage;  added many
    more patient
    * hand gestures data collection
    * gestures work leading to research poster at SOBP
    * Rozenberg, Z., K. Urban, C. Espino, U. Rubin, N. Shokhirev, T.
      Roberts, E. Marsh, F. Postma & D. Brunner. 2019. Quantifying
      stereotypic hand movements in Rett Disorder. 2019 Annual Meeting -
      SOBP’s 74th Annual Scientific Conference. “Innovations in Clinical
      Neuroscience: Tools, Techniques and Transformative Frameworks”.
    * National Organization for Rare
      Disorders (NORD): 2019 Rare Diseases and Orphan Products
      Breakthrough Summit in Washington, D.C.; October 21-22,
      2019Quantifying Stereotypic Hand Movements in Rare Disorders using
      IMU Sensors and Deep Learning AlgortihmsZ. Rozenberg, K. Urban, C.
      Espino, U. Rubin, N. Shokhirev, T. Roberts, E. Marsch, F. Postma, D.
      Brunner

  \item\leftandright
    {\textbf{Explored wearables}}
    EverSleep, JawBone, Biovotion, BioBeats, EarlySense Live, Fitbits,
    apple watch, actigraph, Sensoria, Emfit, MbientLabs, Empatica
  \item\leftandright
    {\textbf{Investigated utility of Wearables data for Parkinson's
    Disease}}
    * Empatic E4, Physiolog, MJFF
  \item\leftandright
    {\textbf{Partnered with clinical care network}}
    * helped develop project map for suicide/self-harm research and detection
    efforts; * developed causal and predictive models for patient
    dropout
    * began working on TEDS-D SUD data set for dropout model
    * how to deal w/ missing values in predictive models
    * developed predictive and causal goals with CVN statistician* how
    to measure model performance on model predictions that are intended
    to be undone?* how to understand causal effects?

  \item\leftandright
    {\textbf{PTSD Landscaping and Knowledge Mapping}}
    ...to help guide adaptive clinical trials...
  \item\leftandright
    {\textbf{Database Evaluation and Implementation for
    Wearables-to-Biological-Phenomenon Knowledge Graph}}
    {\small{Neo4j, MongoDB, MySQL, Postgres, etc}} \\  
    * Reviewed the functionality and pros/cons of graph (Neo4j,
      AgensGraph, AWS Neptune), document (MongoDB), relational (MySQL,
      Postgres, TimeScaleDB), and key-value (Redis, DynamoDB) databases,
      as well as enhanced AWS cloud offerings, such as RDS, Athena, and
      Aurora
    * Designed several iterations of schema (e.g., entity relationship
      diagrams, document-oriented schema, and graph models)
    * Identified graph databases as providing the best support for 
      many-to-many/any-to-any relationships inherent in our concept
      mappings
    * Demonstrated the simplified nature of querying via Cypher from Neo4j 
    * Mapped queries of interest to Cypher...
    * data dictionaries
  \item\leftandright
    {\textbf{Landscaping and Literature Review: Suicide and Self-Harm Predictors and Risk Factors}}
    {\small{}} \\  
    * Developed domain expertise in suicidal ideation and behaviors
    * Synthesized and reported on the state-of-the-art modeling efforts from several
      siloed communities of research, including efforts involving
      wearable and mobile phone sensors, electronic health records,
      brain imaging, paralinguistics, natural language processing, 
      neurocognitive tests, clinical instruments
      (questionnaires) used for screening, diagnosing, and monitoring
      patients 
    * * wrote suicide review paper (then let it collect dust)
    * Suicide report to white house (Request for Information, RFI);
    review of suicide detection methods (data types (brain imaging,
    scales, neurocog tests, EHRs, mobile phones / wearables, etc),
    models)  (In response to:  Executive Order (EO) 13861 (the
    President’s Roadmap to Empower Veterans and End a National Tragedy
    of Suicide or PREVENTS), On signed on March 5, 2019)
\end{itemize*}
