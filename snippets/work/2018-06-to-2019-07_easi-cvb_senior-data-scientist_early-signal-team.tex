\leftandright
  {Senior Data Scientist (Early Signal Team)}
  {July 2018 - Jun 2019} \\
%\begin{itemize*}
  %\item Wearables-based Modeling of Clinical Risk Scales (Clinical Partnership)
  %\item Clinical Dropout Modeling (Clinical Partnership)
  %\item Early Signal Platform - Knowledge Grap
  %\item Stereotypy and Gesture/Activity Recognition in Rett Syndrome
  %\item Funding Acquisition Suppor
  %\item\leftandright
  %  {\textbf{Wearables-based Modeling of Clinical Risk Scales (Clinical Partnership)}} {} \\  
    % Description
    %Designated as a partnership liaison with Cohen
    %Veterans network (CVN), a non-profit clinical care group (independent
    %of CVB, but with same primary source of funding). Responsible 
    %for thought leadership and R\&D at the intersection of wearables,
    %machine/deep learning, and mental healthcare; and managing the
    %ideation and development of a viable, beneficial collaboration between the two
    %organizations (in spite of divergent goals and interests).
    %Activities included regular brainstorming workshops, roadmapping,
    %literature reviews, market/industry landscaping (e.g.,
    %available iPhone/Android apps dedicated to mental health detection
    %task), and data science initiatives.  Literature reviews included
    %the synthesis of the current state of research in prediction/classification 
    %modeling of mental health phenomena (e.g., suicidality, 
    %risk of self-harm, depression, anxiety, and stress) across a variety
    %of data collection domains, including wearable/smarthphone sensors,
    %brain imaging, clinical scales, neurocognitive testing, and natural 
    %language processing (suicide notes, sleep journals, social media);
    %the work especially focused on the challenges and nuances of suicide
    %prediction (e.g., comparative review of risk factors contrasting 
    %suicidal ideators, attempters, and completers).      
  %\item\leftandright
  %  {\textbf{Clinical Dropout Modeling (Clinical Partnership)}} {} \\  
    %A continuation of CVB/CVN partnership management, but with a change
    %in direction to be better aligned with the goals and interests of
    %both parties in the collaboration (based on my work and recommendations 
    %from the earlier effort). This pivot was 2-fold. For one, directly 
    %predicting an impending suicide via standard measures, such as the
    %use of clinical scales, is shown in meta-analyses to be no 
    %better than chance, which gives little direction novel measures, 
    %such as those based on wearables, especially without any large-scale 
    %datasets to mine. Secondly, CVN did not collect this type of data
    %and was hesitant to begin doing so. Working with a new lead at CVB, 
    %we were able to pivot the collaboration's focus to the use of
    %electronic health records (EHRs) in the modeling of more practical targets,
    %such as dropout from treatment. We considered a 2-prong approach
    %(using machine learning and causal inference models in parallel) 
    %that we hoped would help cope with the remaining issue with ``intent
    %to intervene'' models, where model predictions trigger interventions
    %that prevent a straightforward assessment of the model's predictions
    %in deployment. I was responsible for the machine learning side of
    %the collaboration and developed several mini-project prototypes on
    %publicly-available dropout datasets, which aimed to demonstrate the
    %utility of ``blackbox'' models and ease concerns about
    %interpretability.  These projects included topics such as: 
    %``Prediction vs Explainability vs Interpretability,'' ``Categorical
    %variables and Random Forests,'' ``Missing Data and Informative
    %Missingness,'' and ``Blackbox Interpretability Techniques''
    %(covering several types of techniques at both the aggregate and
    %per-decision level).

  %\item\leftandright
  %  {\textbf{Early Signal Platform - Knowledge Graph}} {} \\  
    %As the team developed literature reviews covering many different
    %areas of mental health and wearable sensors, we wanted a way to
    %combine and synthesize the lessons learned. I worked on developing
    %schemas that related the sensor used in wearables to symptoms and
    %other measurable biology/physiology-related phenomena via composite,
    %hierarchical, and/or recursive trajectories through a sequence of
    %algorithms and data signatures

    %| 2018-10-05 | Presentation | EaSiEco (Graph)  | Several Wearable
    %Table Schemas Already In-Use      | 1, 2.2 |
    %| 2018-10-12 | Presentation | EaSiEco (Graph)  | Database Landscape
    %and Data Modeling               | CVB, Manager (Dani) update |
    %| 2018-10-23 | Presentation | EaSiEco (Graph)  | Neo4j-Pricing
    %| Internal use, recommendations |
    %| 2018-10-30 | Presentation | EaSiEco          | Neo4j-Cypher
    %Translations of Natural Language Queries | Prelim slides/tables for
    %internal team |
    %| 2019-02-08 | Presentation | EaSiEco (Graph)  | EaSi Ecosystem Info
    %Deck                           | 1-on-1s w/ Zhanna; later on at
    %CVB-WWG w/ Deepti |
    %| 2019-05-01 | Conference Poster | Rett        | Quantifying
    %Stereotypic Hand Movements in Rett Syndrome | Society of Biological
    %Psychiatry (SOBP, 74th Annual Meeting) |


  %\item\leftandright
  %  {\textbf{Stereotypy and Gesture/Activity Recognition in Rett Syndrome}} {} \\  

  %\item\leftandright
  %  {\textbf{Funding Acquisition Support}} {} \\  
    %Contributed a summary of the state of predictive analytics in
    %suicide research for grants submitted to funding agencies, such
    %as the Office of Naval Ressearch. Contributed slides and analyses of 
    %sample wearables data provided to the Early Signal team from the 
    %Michael J. Fox Foundation (MJFF) for use in a pitch to MJFF
    %that helped establish CVB/EaSi funding over the next several years. 
%\end{itemize*}
