\leftandright{WWE (World Wrestling Entertainment)}{Stamford, CT \textbullet\, October
2016 -- July 2018} \\  
\leftandright{\emph{Data Scientist, Advanced Analytics Team}}{} \\
  Lead on numerous business-driven analytics research and
  predictive modeling projects for WWE Network -- an
  over-the-top (OTT) content distribution network 
  (described as ``Netflix for Wrestling'') with nearly
  2 million daily active users, resulting in 1-3 million new rows of viewership/behavioral 
  data in an AWS Redshift database everyday
\begin{itemize*}
  \item\leftandright
    {\textbf{Customer Behavioral Models}}
    {\small{(Python, Sklearn, TensorFlow, Keras, AWS
    EC2/Redshift, SQL)}} \\  
    Development of automated reporting and interactive dashboard 
    pipelines for regular distribution of live or recent subscriber 
    and business health metrics to key WWE busines stakeholders
    across verticals, as well as at the executive level (Vince Mcmahon,
    Paul "Triple H" Levesque); this included the development of
    data processing and machine learning pipelines
    (preprocessing, feature selection, dimensionality reduction,
    model selection, etc) for the training and deployment of prediction and classification
    models (random forests, deep multilayer perceptrons,
    CNNs), which provided insights for downstream decision support.  Behavioral targets include
    customer segmentation, churn, winback, and lifetime value.        
  \item\leftandright
    {\textbf{Wrestler Recognition Network}}
    {\small{(Python, TensorFlow, Keras, AWS, EC2, GPU)}} \\  
    Development of wrestler image recognition algorithm (transfer learning,
    convolutional neural networks) that could be used to tag video
    and image content.
  \item\leftandright
    {\textbf{Seasonal Subscriber Analysis}}
    {\small{(R, Python, FFT)}} \\  
    Spectral analyses of customer churn/winback behaviors 
    to enhance customer segmentation efforts, help advise email campaigns, and
    improve churn/winback forecasting 
  \item\leftandright
    {\textbf{Revenue Attribution Across Content}}
    {\small{(R, tidyverse, SQL, Redshift)}} \\  
    Development of ensemble approach to revenue attribution models for WWE Network content
    driven by viewership behaviour at the customer level (analysis of
    over a billion rows of data in Redshift)
  \item\leftandright
    {\textbf{Miscellaneous Projects}}
    {\small{(R, tidyverse, SQL, Redshift, Tableau)}}
    \begin{itemize*}
      \item Fusion of fan panel survey response data with customer accounts on the
        WWE Network to better \textbf{quantify the relationship between
        subjective responses and objective viewership data}
      \item \textbf{Sentiment analysis} of customer cancellation surveys
        and social media content to uncover resolvable issues and inform on
        how to better serve our customer base
      \item Design of \textbf{interactive dashboards in Tableau} to help interested
        parties explore our data assets
    \end{itemize*}
\end{itemize*}

%\vspace{-0.3em}

\leftandright{\emph{Data Scientist, Content Analytics Team}}{} \\
  Lead on researching and developing multiple data engineering and
  automation efforts to establish consistent and efficient data collection
  across WWE's digital assets, including a variety of YouTube channels,
  100's Facebook Pages and Twitter accounts, and live streaming across
  multiple online platforms.
\begin{itemize*}
  \item\leftandright
    {\textbf{YouTube Assets}}
    {\small{(Python, Google Client API, YouTube Data API, YouTube
    Reporting API, Redshift, SQL, Selenium, Hive, Presto, Tableau)}}
    \begin{itemize*}
      \item \textbf{Research into and documentation} of several YouTube APIs
        (Google Client, YouTube Data, YouTube Analytics, YouTube
        Reporting), establishing that 3rd party services were capturing 
        only a fraction of the available content and viewership data from 
        our YouTube assets.  
      \item \textbf{Persuasion for buy-in from leadership}:  3rd party
        consultants responsible for the collection/warehousing of
        our YouTube assets argued they were already 
        collecting everything possible; thanks to my
        research and documentation, key decision makers supported my 
        plan to revamp the YouTube data pipeline.
      \item \textbf{Development and deployment} of an        
        automated YouTube data collection and warehousing pipeline 
        over a variety of WWE-owned channels, increasing the daily volume of
        valuable YouTube  more than tenfold, resulting in significant 
        cost savings by obviating need for 3rd party consultants
      \item \textbf{Design and development} of a variety of
        live Tableau dashboards and automated emails to the appropriate
        stakeholders.
      \item \textbf{Shorter Time to Insight}: YouTube takes its viewership metrics
        (e.g., number of view, likes, etc) very seriously and has many
        algorithms in place to detect and correct for fraudulent sources
        (robots, automated page refreshes, etc).  In this spirit, there
        is a 3-day lag on any officially published data.  However, our
        marketers, advertisers, and other leadership want insights and
        intuition on content performance immediately.  By using Selenium
        for webscraping, I was able to extract ``unofficial'' YouTube
        viewership data with no lag -- especially important on the day a
        video is live or published. This is something the data vendor was
        not getting for WWE.
      \item \textbf{Hive/Presto vs Redshift}: Worked with data
        engineering team on a Hive/Presto solution on top of S3 to
        lessen our dependence on Redshift, which is more expensive, 
        especially for the amount of data we were ingesting
    \end{itemize*}
  \item\leftandright
    {\textbf{Facebook/Instagram Assets}}
    {\small{(Python, Facebook Graph API, Selenium, Redshift, SQL)}}
    \begin{itemize*}
      \item My success evaluating and re-engineering our YouTube
        data pipeline motivated us to take a look at other services
        provided by 3rd party vendors starting with Facebook (WWE owns
        and manages 100's of Facebook Pages, several dozen Instagram
        accounts, as well as several Facebook
        exclusive web series that air live weekly)
      \item Researched and documented various Facebook APIs and
        technologies, including the various web-based data portals,
        GraphQL, and the Facebook Graph API.
      \item By learning how to use the Graph API, I was able to 
        query, pull, and storing all Page, Post, and Video Insights on a
        daily basis, which required flattening JSON files into a tabular
        form amenable to SQL queries in Hive or Redshift
      \item By comparing my daily pull with our vendors, I identified that 
        the vendors were only providing a limited subset of
        the Page , Post, and Video Insights in our daily Facebook load
        (and no Instagram data)
      \item I also learned how track live videos at a high frequency,
        thus providing data for live videos on our various Facebook
        Pages, as well as live webisodes streaming on Facebook Watch --
        data the vendors were not able to provide 
    \end{itemize*}
  \item\leftandright
    {\textbf{Multi-Platform Real-Time Dashboard for Live Events}}
    {\small{(Python, YouTube APIs, Facebook Graph API, Selenium, Redshift, SQL)}}
    \begin{itemize*}
      \item Lessons learned from YouTube and Facebook pipeline
        development motivated the development of a live Tableau
        dashboard that could be published within minutes of any live
        airing of a "kickoff show" (aired on YouTube, Facebook, Twitter,
        the WWE Website, and the WWE Network) before a main wrestling event (aired
        only the WWE Network) 
      \item This required an immediate, automated capture of viewership metrics across 
        all platforms the event aired live on, replacing a highly manual
        process performed by several employees over multiple
        departments, resulting in a 12-24 hour latency
      \item Required the automation of headless web browsers (using Selenium, Chrome,
         BeautifulSoup, and Cron) to scrape JavaScript-heavy Twitter and WWE
         websites 
    \end{itemize*}
  \item\leftandright
    {\textbf{R Packages for Automated Reporting}}
    {\small{(R, tidyverse, Redshift, SQL)}} \\
    Developed R packages that the team could use to automate
    various reporting responsibilities
\end{itemize*}

