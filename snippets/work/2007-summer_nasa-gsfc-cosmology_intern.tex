\leftandright{NASA Goddard Space Flight Center }{ }
\leftandright{\textcolor{white}{space} }{Greenbelt, MD \textbullet\, June -- August 2007}\\  
\vspace{-0.8em}
\textit{Intern, Observational Cosmology Laboratory} 
\begin{itemize*}
  \item The Absolute Spectrum Polarimeter [ASP] was a mission concept
    under study at GSFC's Observational Cosmology Lab with the goal of
    measuring the polarization of the Cosmic Microwave Background (CMB) 
    via a polarizing Michelson interferometer sensitive enough to detect
    B-mode polarization, if it exists, which is theorized to hold a unique
    signature indicative of a primordial backdrop of gravitational waves 
    predicted by inflationary theory.
  \item Responsible for reviewing the theory of general relativity and
    cosmological inflation theory, as well as conducting self-guided
    literature reviews into prior spacecraft missions and research programs
    dedicated to designing and operationalizing experimental/observational
    approaches to measuring associated phenomena, such as gravitational waves
    and anisotropies in the CMB, and detecting the presence of predicted
    data signatures.
  \item Implementation of a forward model that simulates the
    input expected by the ASP sensors (e.g., sky polarization maps
    modeling an anisotropic CMB), the corresponding sensor response
    data, and the anticipated data processing and transformation steps
    necessary for extracting the Stokes parameter representation of
    the CMB polarization (as a function of frequency and direction),
    while accounting for potential confounding sources of polarization
    (such as dust in the galactic foreground).
  \item Programming tools utilized in the development of this
    software included IDL, MatLab, Bash, and misc command line tools
\end{itemize*}

