%========================================================
% FULL-DOCX MERGED VERSION (WINNERS)
%========================================================
\leftandright{NJIT Department of Physics}{Newark, NJ \textbullet\, 
August 2008 - August 2010} \par
\textit{Research Assistant, Full Time}
\vspace{-0.8em}
\begin{itemize*}
  % 1: MODIFIED
  \item Developed data processing and modeling pipelines for
    surveillance of the geomagnetic field in targeted regions (such as
    the southern polar cap) and globally
  % 2: MODIFIED
  \item Processing pipelines first ingest, QC, and harmonize sensor time series data
    from one or several multi-channel sensors (triaxial fluxgate 
    magnetometer, 1 Hz), e.g., from a network of magnetometers 
    distributed throughout Antarctica
  % 3: MODIFIED
  \item Processed data is fed into the appropriate
    analytical/algorithmic pipeline(s) composed of parallel and/or
    sequential submodules where each submodule applies a specified type 
    (e.g., linear/nonlinear, parametric/nonparametric, deterministic/stochastic
    and/or learning/rules-based) of signal processing technique or transformation 
    for purposes such as sensor fusion/integration, feature engineering/extraction,
    parameter estimation, and/or dimensionality reduction
  % 4: MODIFIED
  \item Trained and/or metric/rules-based models (e.g., event
    detection/classification) then leverage the transformed data representation(s)
    to comprehensively monitor, track, or further model polar cap structure,
    ionospheric currents, geomagnetic activity, and solar wind inputs
  % MOD
  \item The data processing, modeling, and visualization pipeline
    elements were implemented using R, MatLab, IDL, and/or Bash
    (depending on task and project).
\end{itemize*}
