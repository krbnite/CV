%========================================================
%  WORK / RESEARCH   EXPERIENCE
%========================================================
\vspace{0.2cm}
\ressection{Work and Professional Research Experience}

\leftandright{Department of Physics, NJIT}{Newark, NJ \textbullet\, January
2012 -- present} \\  
\vspace{-0.8em}
\textit{Research Scientist, Ph.D. Candidate. Advisors: Andrew Gerrard,
Louis Lanzerotti, Denis Blackmore} 
\begin{itemize*}
  \item 
    Through the combination of data/observationally-driven research and deep
    physical understanding of magnetohydrodynamics, explained
    decades-old problem which prevented proper understanding of the geomagnetic field's
    complex structure in the polar regions. The research involved
    stochastic analysis; development of novel visualization and
    analysis techniques; collection, harmonization, and normalization of
    100's of data sets distributed across the Earth and solar system
    to model and understand the linear/nonlinear drivers of complex,
    physical systems.
  \item The product development included statistical and physical model
    selection, new interpretations of results commonly understood as
    anomalies found in the empirical spatial distribution, hypothesis
    testing, and development of dynamic correlation metrics which verified 
    time-varying causal relationships, thus verifying the 
    contrary hypothesis to previous understanding of the subject matter.
\end{itemize*}


%-------------------------------------------------------------
\leftandright{Department of Mathematical Sciences, NJIT}{Newark, NJ \textbullet\, 2008 -- 2012}\\  
\vspace{-0.8em}
\textit{Research Scientist, Professor Denis Blackmore} 
\begin{itemize*}
  \item
    Investigated sources of chaos in a 
    dimensionally-reduced parameter space of a discrete-dynamical system via
    \href{http://kevin-urban.com/video/R__gamma-var_restitution-0.8.gif}{Poincare
    map simulations in R} 
  \item
    Numerically modeled and contrasted several representations of granular fluid systems
    \href{http://www.kevin-urban.com/images/gran_tapping-pics.png}{in
    MatLab}. 
  \item Worked with numerically stiff difference equations requiring a stability
    analysis to understand how to minimize the propagation of numerical errors, which otherwise
    prevent proper quantitative analysis and visualization of the underlying dynamical system.
  \item
    \href{http://msp.org/jomms/2011/6-1/jomms-v6-n1-p06-s.pdf}{Published
    papers} on
    \href{http://www.sciencedirect.com/science/article/pii/S0167278914000189}{granular
    fluid systems} and 
    \href{http://www.icmp.lviv.ua/journal/zbirnyk.64/}{fractional
    dynamics} 
  \item
    Gained knowledge in advanced mathematical disciplines such as
    dynamical systems, topological analysis, manifold theory, abstract
    algebra, stochastic calculus
  %\item Gained practical experience using the R programming language
\end{itemize*}

% \vspace{0.25cm}
% \leftandright{Department of Physics and Astronomy, Siena
% College}{Loudonville, NY \textbullet\, August 2011 -- January 2012}\\  
% \vspace{-0.8em}
% \textit{Research Scientist, Contractor} 
% \begin{itemize*}
%   \item
%     Continued development of
%     \href{http://kevin-urban.com/images/IDL__Pretty-PSD.pdf}{time series, spectral, 
%     and regression data analysis techniques} in effort to delineate open
%     and closed geomagnetic flux using magnetometers situated in and
%     around the southern polar cap
% \end{itemize*}


%-------------------------------------------------------------
\leftandright{NASA/CalTech Jet Propulsion Laboratory}{Pasadena, CA \textbullet\, Summer 2011}\\  
\vspace{-0.8em}
\textit{Intern, Trajectory Optimization / Design} 
\begin{itemize*}
  \item
    Participated in the development of a full-fledged spacecraft mission to
    the Trojan asteroids of Jupiter
    (from % responding to an Announcement of Opportunity [AO]
    establishing and prioritizing science goals, to optimizing the science-engineering-financial
    parameter space, to the written proposal and presenting our mission design to the NASA review
    board)
  \item 
    Gained an appreciation of rapid product development via concurrent engineering
  \item Gained understanding of the intricate interplay between the
    various engineering system designs, science goals, timeline
    requirements, and budget constraints.
  %\item
  %  Gained understanding of the interplay between the trajectory design
  %  and other considerations, such as launch
  %  windows, budget time lines, a multi-component propulsion system,
  %  and payload
  %\item 
  %  Helped develop the science time line and traceability matrix for the
  %  baseline (desired) and threshold (minimum requirement) missions

\end{itemize*}
   
%-------------------------------------------------------------
\leftandright{Department of Physics, NJIT}{Newark, NJ \textbullet\, 
August 2008 - August 2010}\\  
\vspace{-0.8em}
\textit{Research Scientist, M.S. Student. Advisors: Andrew Gerrard,
Louis Lanzerotti} 
\begin{itemize*}
  \item
    Analyzed multi-channel instrument data from a spatially-distributed
    network of automated observatories in Antarctica.
  \item Developed data metrics and associated classification scheme for
    events of interest in the data sets.
  \item Demonstrated the feasibility of a real-time
    detection/classification scheme in maintaining surveillance on
    local, regional, and global-scale geomagnetic events of interest.
  \item Gained practical experience in various data analysis techniques,
    such as digital signal processing, time series analysis, spectral analysis,
    statistical methods, and regression.
  \item In-depth experience visualizing and analyzing data in R, MatLab,
    and IDL
  \item Published Master's thesis
    %: Synoptic variability of a CIR-driven
    %open-closed boundary during solar minimum.
\end{itemize*}

%-------------------------------------------------------------
\leftandright{NASA Goddard Space Flight Center }{ }
\leftandright{\textcolor{white}{space} }{Greenbelt, MD \textbullet\, June -- August 2007}\\  
\vspace{-0.8em}
\textit{Intern, Observational Cosmology Laboratory} 
\begin{itemize*}
  \item Developed software implementing a Stokes parameter analysis to
    run on simulation data for the proposed Absolute Spectrum
    Polarimeter [ASP] — an instrument designed to detect B-mode
    gravitational waves, which would provide evidence for Einstein’s
    theory of general relativity and cosmological inflationary theory
  \item Gained practical experience in IDL, MatLab, and shell scripting
  % \item Studied cosmology as it related to the mission goals of ASP
\end{itemize*}


%-------------------------------------------------------------
\leftandright{NASA Goddard Space Flight Center}{Greenbelt, MD \textbullet\, June -- August 2006}\\  
\vspace{-0.8em}
\textit{Intern,  Heliophysics Division}
\begin{itemize*}
  \item Worked extensively in the UNIX programming environment, learning and
    developing Python, HTML, CSS, JavaScript, and PHP code
  \item Developed web content  % for the International Heliophysical Year [IHY]
\end{itemize*}

