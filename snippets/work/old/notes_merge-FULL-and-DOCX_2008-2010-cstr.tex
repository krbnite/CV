%========================================================
%  FULL VERSION
%========================================================
\leftandright{NJIT Department of Physics}{Newark, NJ \textbullet\, 
August 2008 - August 2010} \par
\textit{Research Assistant }
\vspace{-0.8em}
\begin{itemize*}
  \item Led project demonstrating the feasibility of a real-time
    surveillance, detection, classification, and prediction of 
    local, regional, and global-scale geomagnetic events of interest
  \item
    Analyzed multi-channel instrument data from a spatially-distributed
    network of automated observatories in Antarctica.
  \item Developed data metrics and associated classification scheme for
    events of interest in the data sets.
  \item In-depth experience visualizing and analyzing data in R, MatLab,
    and IDL
  \item In-depth experience in various data analysis techniques,
    such as digital signal processing, time series analysis, spectral analysis,
    statistical methods, modeling, and regression.
  \item Hardware-software interfacing of instrumentation for data
    collection
  \item Published Master's thesis
\end{itemize*}


%========================================================
%  DOCX VERSION
%========================================================
\leftandright{Department of Physics, NJIT}{Newark, NJ \textbullet\, 
August 2008 - August 2010}\\  
\vspace{-0.8em}
\textit{Research Scientist, M.S. Student. Advisors: Andrew Gerrard,
Louis Lanzerotti} 
\begin{itemize*}
  \item
    Analyzed multi-channel instrument data from a spatially-distributed
    network of automated observatories in Antarctica.
  \item Developed data metrics and associated classification scheme for
    events of interest in the data sets.
  \item Demonstrated the feasibility of a real-time
    detection/classification scheme in maintaining surveillance on
    local, regional, and global-scale geomagnetic events of interest.
  \item Gained practical experience in various data analysis techniques,
    such as digital signal processing, time series analysis, spectral analysis,
    statistical methods, and regression.
  \item In-depth experience visualizing and analyzing data in R, MatLab,
    and IDL
  \item Published Master's thesis
    %: Synoptic variability of a CIR-driven
    %open-closed boundary during solar minimum.
\end{itemize*}


%========================================================
% COMPARE VERSION -> SELECT WINNERS
%========================================================
%% 1
%%%% FULL != DOCX: FULL (slightly better) -> Modified (even better)
  \item Led project demonstrating the feasibility of a real-time
    surveillance, detection, classification, and prediction of 
    local, regional, and global-scale geomagnetic events of interest
  \item Demonstrated the feasibility of a real-time
    detection/classification scheme in maintaining surveillance on
    local, regional, and global-scale geomagnetic events of interest.
  % MODIFIED
  \item Developed data processing and modeling pipelines for
    surveillance of the geomagnetic field in targeted regions (such as
    the southern polar cap) and globally
%% 2
%%%% FULL = DOCX
  \item
    Analyzed multi-channel instrument data from a spatially-distributed
    network of automated observatories in Antarctica.
  % MODIFIED
  \item Processing pipelines first ingest, QC, and harmonize sensor time series data
    from one or several multi-channel sensors (triaxial fluxgate 
    magnetometer, 1 Hz), e.g., from a network of magnetometers 
    distributed throughout Antarctica
%% 3
%%%% FULL != DOCX: MODIFY FOR BETTER NARRATIVE!
  \item In-depth experience in various data analysis techniques,
    such as digital signal processing, time series analysis, spectral analysis,
    statistical methods, modeling, and regression.
  \item Gained practical experience in various data analysis techniques,
    such as digital signal processing, time series analysis, spectral analysis,
    statistical methods, and regression.
  % MODIFIED (ATTEMPT #1)
  \item Gained hands-on, practical experience in time series and spectral 
    analyses of multi-modal sensor datasets (digital signal processing),
    statistical modeling, and basic machine learning techniques 
  % MODIFIED (ATTEMPT #2)
  \item Processed data is then fed into the appropriate
    analytical/algorithmic pipeline(s) composed of parallel and/or 
    sequential submodules which leverage techniques from digital
    signal processing, nonlinear/stochastic time series and spectral
    analyses, etc, for purposes such as sensor fusion/integration, 
    feature engineering/extraction, parameter estimation, and/or 
    dimensionality reduction
  % MODIFIED (ATTEMPT #3 - WINNER)
  \item Processed data is then fed into the appropriate
    analytical/algorithmic pipeline(s) composed of parallel and/or
    sequential submodules where each submodule applies a specified type 
    (e.g., linear/nonlinear, parametric/nonparametric, deterministic/stochastic
    and/or learning/rules-based) of signal processing technique or transformation 
    for purposes such as sensor fusion/integration, feature engineering/extraction,
    parameter estimation, and/or dimensionality reduction
%% 4
%%%% FULL = DOCX: MODIFY FOR COMPELLING NARRATIVE!
  \item Developed data metrics and associated classification scheme for
    events of interest in the data sets.
  % MODIFIED
  \item Trained and/or metric/rules-based models (e.g., event
    detection/classification) then leverage the transformed data representation(s)
    to comprehensively monitor, track, visualize, or further model polar cap structure,
    ionospheric currents, geomagnetic activity, and solar wind inputs
%% 5
%%%% FULL = DOCX: MOD!
  \item In-depth experience visualizing and analyzing data in R, MatLab,
    and IDL
  % MOD
  \item The data processing, modeling, and visualization pipeline
    elements were implemented using R, MatLab, IDL, and/or Bash
    (depending on task and project).
%% 6
%%%% FULL != DOCX: not in DOCX (typically apart of the undergrad description)
  \item Hardware-software interfacing of instrumentation for data
    collection


%%%%%%%%%%%%%%
%========================================================
% FULL-DOCX MERGED VERSION (WINNERS)
%========================================================
\leftandright{NJIT Department of Physics}{Newark, NJ \textbullet\, 
August 2008 - August 2010} \par
\textit{Research Assistant, Full Time}
\vspace{-0.8em}
\begin{itemize*}
  % 1: MODIFIED
  \item Developed data processing and modeling pipelines for
    surveillance of the geomagnetic field in targeted regions (such as
    the southern polar cap) and globally
  % 2: MODIFIED
  \item Processing pipelines first ingest, QC, and harmonize sensor time series data
    from one or several multi-channel sensors (triaxial fluxgate 
    magnetometer, 1 Hz), e.g., from a network of magnetometers 
    distributed throughout Antarctica
  % 3: MODIFIED
  \item Processed data is then fed into the appropriate
    analytical/algorithmic pipeline(s) composed of parallel and/or
    sequential submodules where each submodule applies a specified type 
    (e.g., linear/nonlinear, parametric/nonparametric, deterministic/stochastic
    and/or learning/rules-based) of signal processing technique or transformation 
    for purposes such as sensor fusion/integration, feature engineering/extraction,
    parameter estimation, and/or dimensionality reduction
  % 4: MODIFIED
  \item Trained and/or metric/rules-based models (e.g., event
    detection/classification) then leverage the transformed data representation(s)
    to comprehensively monitor, track, or further model polar cap structure,
    ionospheric currents, geomagnetic activity, and solar wind inputs
  % MOD
  \item The data processing, modeling, and visualization pipeline
    elements were implemented using R, MatLab, IDL, and/or Bash
    (depending on task and project).
\end{itemize*}
