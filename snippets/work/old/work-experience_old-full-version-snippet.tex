%========================================================
%  WORK / RESEARCH   EXPERIENCE
%========================================================
\vspace{0.2cm}
\ressection{Work and Professional Research Experience}

\leftandright{WWE, Advanced Analytics Team}{Stamford, CT \textbullet\, October
2016 -- Present} \\  
\vspace{-0.8em}
\textit{Data Scientist} 
\begin{itemize*}
  \item Lead on numerous business-driven analytics research and
    predictive modeling projects for WWE Network -- an
    over-the-top (OTT) content distribution network  that can be
    described as ``Netflix for Wrestling'' and currently supports nearly
    2 million active users, generating 1-3 million new rows of viewership/behavioral 
    data in Redshift database every day
  \item Experimentation and development of machine learning pipelines
    in Python / Scikit-Learn (preprocessing, feature selection, dimensionality reduction) and 
    deep neural network models in Keras and TensorFlow 
    on AWS EC2 x2.2xlarge (GPU instance) for a handful of projects related customer segmentation, churn, 
    and lifetime value (multilayer perceptrons, autoencoders, recurrent
    neural networks) 
  \item Automation of headless web browsers (using Selenium,
    BeautifulSoup, and Cron) and web APIs (YouTube Reporting, Analytics,
    Data; Google Client; FaceBook Graph) to aid in the collection and
    ingestion of important data sets at desirable cadences
  \item  Development of wrestler image recognition algorithm (transfer learning,
    convolutional neural networks)
  \item  Spectral analyses of customer churn/winback behaviors 
    to enhance customer segmentation efforts, help advise email campaigns, and
    improve churn/winback forecasting 
  \item Development of ensemble approach to revenue attribution models for WWE Network content
    driven by viewership behaviour at the customer level (analysis of
    over a billion rows of data in Redshift)
  \item  Fusion of fan panel / survey response data with customer accounts on the
    WWE Network to better quantify the relationship between
    subjective responses and objective viewership data
  \item Natural language processing (sentiment analysis) of customer cancellation surveys
    and social media content to uncover resolvable issues and inform on
    how to better serve our customer base
  \item  Automation of report generation and analytic work flows
    through the development of project-agnostic and project-specific R packages 
    and python scripts
    (data extraction, cleansing, transformation, analysis, reporting,
    website scraping)
  \item  Decision tree analyses to help understand and segment
    customers by WWE content and character preferences
  \item Design of interactive dashboards in Tableau to help interested
    parties explore our data assets
\end{itemize*}


%-------------------------------------------------------------
\leftandright{NJIT Center for Solar-Terrestrial Research}{Newark, NJ \textbullet\, January
2012 -- May 2016} \\  
\vspace{-0.8em}
\textit{Research Scientist} 
\begin{itemize*}
  \item Fusion and harmonization of data sets from a ground-based,
    globally-distributed network of instruments; high-altitude 
    satellites, such as NOAA's DMSP fleet and NASA's Van Allen
    Probes; and interplanetary spacecraft (NASA's ACE)
  \item Provided new, empirical insights and interpretation of
    decades-old problem concerning the geomagnetic field's complex
    structure in Earth's polar regions
  \item Developed prediction scheme to forecast intensity and
    spatial-temporal distribution the magnetic weather
    in Earth's deep polar cap region using spacecraft upstream of Earth in the
    solar wind (NASA's ACE) 
  \item Inverted the prediction scheme to leverage it as a remote-sensing technique 
    such that data from ground-based magnetometers in Antarctica can
    be used to infer space weather parameters in near-Earth space and,
    if necessary, could essentially replace in-situ spacecraft for certain measurements
  %\item Implemented robust, non-parametric methods to characterize empirical distributions using as
  %  few assumptions as possible
  \item Created innovative visualization and analysis techniques
    to better understand and transform our understanding of the 
    Earth magnetic geography and its dynamic response to varying
    space weather conditions
  \item  Supported myriad data analysis efforts and projects as a member
    of the RBSPICE instrument team for NASA's Van Allen
    Probes mission, which has been essential in understanding and
    forecasting hazardous conditions of Earth's radiation belt
    environment
  \item Developed software packages in Bash, R, MatLab, and IDL to enable efficient, 
    streamlined analyses and reporting for a range of ongoing projects 
  \item Published first-author publication, PhD dissertation, and
    currently have several papers in development
% \item Through the combination of data/observationally-driven research and deep
%   physical understanding of magnetohydrodynamics, explained
%   decades-old problem that prevented proper understanding of the geomagnetic field's
%   complex structure in Earth's polar regions 
%   \textsuperscript{\href{http://www.slideshare.net/inverseuniverse/polarcappower}
%   {[link]}}. The research involved stochastic analysis; development of
%   novel visualization and analysis techniques
%   \textsuperscript{\href{http://www.slideshare.net/inverseuniverse/spectralproducts-64254698}{[link]}};
%   %IDL__8day_multi-mag-spectra.pdf}{example}};
%   collection, harmonization, and normalization of 100's of data sets 
%   \textsuperscript{\href{http://kevin-urban.com/video/R__20130302_trifecta.mp4}{[link]}}
%   distributed across the Earth and solar system to model and
%   understand the linear/nonlinear drivers of complex, physical
%   systems.
% \item The product development included statistical and physical model
%   selection, new interpretations of results commonly understood as
%   anomalies found in the empirical spatial distribution, hypothesis
%   testing, and development of dynamic correlation metrics which verified 
%   time-varying causal relationships 
%   \href{http://kevin-urban.com/images/dynamic-correlograms.pdf}
%   {\textsuperscript{[link]}}, thus verifying the 
%   contrary hypothesis to previous understanding of the subject matter.
\end{itemize*}



%-------------------------------------------------------------
\leftandright{NJIT Department of Mathematical Sciences}{Newark, NJ \textbullet\, 2008 -- 2012}\\  
\vspace{-0.8em}
\textit{Research Scientist, Mathematical Modeling} 
\begin{itemize*}
  \item Developed dynamical system models for granular fluid systems
  \item Simulated and visualized the dynamical models in MatLab and R
    %\href{http://www.kevin-urban.com/images/gran_tapping-pics.png}
    %{\textsuperscript{\tiny{[link]}}} 
  \item Investigated sources of chaos and instability in such systems
    by projecting them into a dimensionally-reduced parameter space 
    and coding simulations of their Poincare maps in R 
    %(e.g., via \href{http://kevin-urban.com/video/R__gamma-var_restitution-0.8.gif}{Poincare
    %map simulations in R \textsuperscript{\tiny{[link]}}})
  \item Led effort in understanding and controlling the propagation of numerical errors, 
    which otherwise prevent proper quantitative analysis and
    visualization of the underlying dynamical system
  \item Co-authored four peer-reviewed papers numerically modeling,
    visualizing, and analyzing granular fluid systems
  \item Led research on potential engineering applications of fractional calculus and
    co-authored peer-reviewed publication 
  \item Explored real-world applications of advanced mathematical disciplines such as
    dynamical systems, topological analysis, manifold theory, abstract
    algebra, stochastic calculus
    % \href{http://msp.org/jomms/2011/6-1/jomms-v6-n1-p06-s.pdf}
    % {\textsuperscript{\tiny{[link1],}}}  
    % \href{http://www.sciencedirect.com/science/article/pii/S0167278914000189}
    % {\textsuperscript{\tiny{[link2],}}} 
    % \href{http://www.icmp.lviv.ua/journal/zbirnyk.64/}
    % {\textsuperscript{\tiny{[link3]}}} 
\end{itemize*}








%-------------------------------------------------------------
\leftandright{NASA/CalTech Jet Propulsion Laboratory}{Pasadena, CA \textbullet\, Summer 2011}\\  
\vspace{-0.8em}
\textit{Intern, Trajectory Optimization / Design} 
\begin{itemize*}
  \item

    Collaborated with a team of engineers and scientists to develop a full-fledged spacecraft mission to
    the Trojan asteroids of Jupiter
    (from % responding to an Announcement of Opportunity [AO]
    establishing and prioritizing science goals, to optimizing the science-engineering-financial
    parameter space, to the written proposal and presenting our mission design to the NASA review
    board)
  \item 
    Gained an appreciation of rapid product development via concurrent
    engineering, and an understanding of the intricate interplay between the
    various engineering system designs, science goals, timeline
    requirements, and budget constraints.
  \item Published peer-reviewed paper documenting lessons learned
  %\item
  %  Gained understanding of the interplay between the trajectory design
  %  and other considerations, such as launch
  %  windows, budget time lines, a multi-component propulsion system,
  %  and payload
  %\item 
  %  Helped develop the science time line and traceability matrix for the
  %  baseline (desired) and threshold (minimum requirement) missions
\end{itemize*}


%-------------------------------------------------------------
\leftandright{Independent Research Consultant}{Clifton, NJ \textbullet\, 
August 2010 - January 2012} \par
\vspace{-0.8em}
\begin{itemize*}
  \item Procured contracts from NJIT's Physics Department and Siena
    College's Department of Physics and Astronomy to 
    continue the development of automated, real-time prediction of geomagnetic parameters
    using geospatially-distributed network of magnetometers
  \item Published peer-reviewed paper documenting proof-of-concept and 
    lessons learned
    \href{http://onlinelibrary.wiley.com/doi/10.1029/2011SW000688/full}
    {proof-of-concept \textsuperscript{\tiny{[link]}}}
    %: Synoptic variability of a CIR-driven
    %open-closed boundary during solar minimum.
\end{itemize*}   


\leftandright{NJIT Department of Physics}{Newark, NJ \textbullet\, 
August 2008 - August 2010} \par
\textit{Research Assistant }
\vspace{-0.8em}
\begin{itemize*}
  \item Led project demonstrating the feasibility of a real-time
    surveillance, detection, classification, and prediction of 
    local, regional, and global-scale geomagnetic events of interest
  \item
    Analyzed multi-channel instrument data from a spatially-distributed
    network of automated observatories in Antarctica.
  \item Developed data metrics and associated classification scheme for
    events of interest in the data sets.
  \item In-depth experience visualizing and analyzing data in R, MatLab,
    and IDL
  \item In-depth experience in various data analysis techniques,
    such as digital signal processing, time series analysis, spectral analysis,
    statistical methods, modeling, and regression.
  \item Hardware-software interfacing of instrumentation for data
    collection
  \item Published Master's thesis
\end{itemize*}




%-------------------------------------------------------------
\leftandright{Center for Solar-Terrestrial Research, NJIT}{Newark, NJ
\textbullet\, January -- August 2008}\\  
\vspace{-0.8em}
\textit{Intern}
\begin{itemize*}
  \item Developed software to interface instrument sensors with an
    analog-to-digital converter
  \item Collected and managed digitized instrument data
  \item Created real-time visualization software (MatLab) for continuous data display
    at local museum in Jenny Jump State Park
  % \item Gained working knowledge of the MatLab programming language
\end{itemize*}


%-------------------------------------------------------------
\leftandright{NASA Goddard Space Flight Center }{ }
\leftandright{\textcolor{white}{space} }{Greenbelt, MD \textbullet\, June -- August 2007}\\  
\vspace{-0.8em}
\textit{Intern, Observational Cosmology Laboratory} 
\begin{itemize*}
  \item Developed software implementing a Stokes parameter analysis to
    run on simulation data for the proposed Absolute Spectrum
    Polarimeter [ASP] --- an instrument designed to detect B-mode
    gravitational waves, which would provide evidence for Einstein's
    theory of general relativity and cosmological inflationary theory
  \item Worked with multiple programming languages, including MatLab,
    Bash (shell scripting), and IDL 
  % \item Studied cosmology as it related to the mission goals of ASP
\end{itemize*}


%-------------------------------------------------------------
\leftandright{NASA Goddard Space Flight Center}{Greenbelt, MD \textbullet\, June -- August 2006}\\  
\vspace{-0.8em}
\textit{Intern,  Heliophysics Division}
\begin{itemize*}
  \item Worked extensively in the UNIX programming environment, learning and
    developing Python, HTML, CSS, JavaScript, and PHP code
  \item Developed web content  % for the International Heliophysical Year [IHY]
\end{itemize*}

