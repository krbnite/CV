\leftandright{\emph{Senior Data Scientist (July 2018 - Jun 2019)}}{} \\
%\begin{itemize*}
%  \item\leftandright
%  \item Co-development, advocacy, and distribution of science-oriented
%    "best practices" for the Early Signal data science team focused on 
%    reducing cognitive overhead and technical debt across project (e.g., 
%    reuseable, customized environment elements, such as a CookieCutter 
%    project template to homogenize layout, baseline Docker images and
%    MiniConda environments, etc)
%  \item Research and writing support for CVB funding initiatives, such as 
%    grant submissions, primarily in the domains of suicidality and PTSD
%  \item Casual collaboration with and mentorship of a junior data
%    scientist interested in diversifying their analytics acumen
%  \item Co-lead on the advancement and support of the 
%    "Early Signal Platform," primarily responsible for the
%    development of wearable-based models capable of detecting,
%    classifying, and further quantifying (e.g., duration, frequency,
%    intensity) an assortment of daily activities, as well as 
%    disease/disorder-specific events, such as symptoms
%    and behaviors, passively and continuous with low-to-no burden
%    for individuals with disruptive
%    neurodegenerative/neurodevelopmental disorders that negatively
%    impact motor function, with the intention that this approach can then
%    help validate and improve treatment and intervention strategies
%    (e.g., is a given medication impacting such measurable quantities in
%    a manner consistent with its purpose?). 
%    \begin{itemize*}
%      \item Development of data processing pipelines to prepare
%        continuously recorded sensor time series data for model
%        training, which -- for example -- requires the 
%        extraction of lab-recorded events  via reference to timestmaped
%        event logs (each event constitutes 20-30 seconds of a subject
%        performing the requested gesture/activit), as well as careful
%        consideration of the windowing strategy (e.g., must be sure to
%        window by timestamp, not index; e.g., advised by data curator
%        that 2-3 seconds at the event boundaries may not actually
%        correspond to the gesture/activity since the timestamps did not
%        always correspond to the exact start/stop times during data
%        collection)
%      \item Careful design of training/validation/test data-splitting
%        strategies dependent on specifics of that dataset,
%        requirements implied by the question the model is intended to
%        address, and other considerations, such as data leakage and
%        accidental training set biases
%      \item Initial prototyping of a gesture/stereotypy recognition
%        model tasked with discerning between nearly 40 classes with
%        high accuracy (~83\%)
%      \item Continued exploration, development, and improvements via
%        hyperparameter tuning in a broadsense, covering degrees
%        of freedom arising at various points along the data processing
%        and modeling pipeline, including signal processing and
%        data-specific considerations (window size, sampling rate, sensor
%        choices, number of sensors, filtering, etc), as well as architectural 
%        decisions determining the overall shape, structure, 
%        composition, and/or topology of the network, further fine-tuned
%        by model internals and learning styles (kernel and stride sizes,
%        regularization and activation specifics, optimizer type, etc),
%        resulting in later stage models capable of discerning
%        between nearly 40 classes at a very high level of
%        accuracy (almost 90\%).
%      \item Modeling contribution to research poster at SOBP
%        (Rozenberg, Z., K. Urban, C. Espino, U. Rubin, N. Shokhirev, T.
%        Roberts, E. Marsh, F. Postma, and D. Brunner. "Quantifying
%        stereotypic hand movements in Rett Disorder." 2019 Annual Meeting -
%        SOBP's 74th Annual Scientific Conference. "Innovations in Clinical
%        Neuroscience: Tools, Techniques and Transformative Frameworks".)
%      \item Modeling contribution to research poster at NORD 
%        (Rozenberg, K. Urban, C. Espino, U. Rubin, N. Shokhirev,
%        T. Roberts, E. Marsch, F. Postma, D. Brunner. "Quantifying
%        Stereotypic Hand Movements in Rare Disorders using IMU
%        Sensors and Deep Learning Algortihms." National Organization
%        for Rare Disorders (NORD): 2019 Rare Diseases and Orphan Products
%        Breakthrough Summit in Washington, D.C.; October 21-22, 2019
%      \item Vision development and brainstorming of operational solutions
%        capable of addressing issues that can arise in deployment for an
%        activity recognition model trained on a labeled, in-lab dataset,
%        but expected to passively and continuously monitor for and recognize
%        the activities it has been trained to detect when deployed "out
%        in the wild." This is the machine learning equivalent of
%        expecting a child to understand the complexity and nuance of
%        moral decision making in adult life: if a model has only ever
%        known a synthetic, oversimplified version of reality depicted by
%        a stringent data collection protocol where each decision its
%        ever had to make, by definition, arises from exactly one of N 
%        orthogonal "pure tone" activities, then one cannot
%        expect it to understand and cope with the complexity,
%        heterogeneity, and nigh-infinitude of potential activities that
%        can arise in a free-living deployment domain -- at least without
%        adopting more sophisticated, adaptive training strategies
%        and deployment-specific customizations.
%        "closed world" dataset 
%      \item Self-guided literature review into techniques from zero-,
%        one-shot, and few-shot learning; domain adaptation; and 
%        transfer learning solutions in general.
%    \end{itemize*}
%
%  \item\leftandright
%    {\textbf{Investigated utility of Wearables data for Parkinson's Disease}}
%    * Empatic E4, Physiolog, MJFF
%
%  \item Project mapping for suicide/self-harm research and detection
%    efforts 
%  \item Problem: How to measure and monitor model performance for a
%    deployed predictive model whose predictions are intended to trigger 
%    interventions, e.g., a suicide prediction model capable of
%    preventing suicides, or a dropout prediction model that helps a
%    clinician to better encourage retention?
%  \item Collaborated with statistician from partner organization to
%    formulate a research trajectory that might benefit from both causal
%    inference and predictive models for patient dropout
%  \item Development of "toy models" to illustrate the strengths and
%    potetial pitfalls of a predictive modeling approach using publicly
%    availble datasets concerned with dropout in clinical trials,
%    in-patient clinical care programs, and out-patient treatment plans 
%  \item Self-guided literature review on treating missing values in
%    predictive models versus inference models, informative missingness,
%    and missingness that contributes to data leakage
%    * developed predictive and causal goals with CVN statistician
%
%
%  \item\leftandright
%    {\textbf{PTSD Landscaping and Knowledge Mapping}}
%    ...to help guide adaptive clinical trials...
%  \item\leftandright
%    {\textbf{Database Evaluation and Implementation for
%    Wearables-to-Biological-Phenomenon Knowledge Graph}}
%    {\small{Neo4j, MongoDB, MySQL, Postgres, etc}} \\  
%    * Reviewed the functionality and pros/cons of graph (Neo4j,
%      AgensGraph, AWS Neptune), document (MongoDB), relational (MySQL,
%      Postgres, TimeScaleDB), and key-value (Redis, DynamoDB) databases,
%      as well as enhanced AWS cloud offerings, such as RDS, Athena, and
%      Aurora
%    * Designed several iterations of schema (e.g., entity relationship
%      diagrams, document-oriented schema, and graph models)
%    * Identified graph databases as providing the best support for 
%      many-to-many/any-to-any relationships inherent in our concept
%      mappings
%    * Demonstrated the simplified nature of querying via Cypher from Neo4j 
%    * Mapped queries of interest to Cypher...
%    * data dictionaries
%  \item\leftandright
%    {\textbf{Landscaping and Literature Review: Suicide and Self-Harm Predictors and Risk Factors}}
%    {\small{}} \\  
%    * Developed domain expertise in suicidal ideation and behaviors
%    * Synthesized and reported on the state-of-the-art modeling efforts from several
%      siloed communities of research, including efforts involving
%      wearable and mobile phone sensors, electronic health records,
%      brain imaging, paralinguistics, natural language processing, 
%      neurocognitive tests, clinical instruments
%      (questionnaires) used for screening, diagnosing, and monitoring
%      patients 
%    * * wrote suicide review paper (then let it collect dust)
%    * Suicide report to white house (Request for Information, RFI);
%    review of suicide detection methods (data types (brain imaging,
%    scales, neurocog tests, EHRs, mobile phones / wearables, etc),
%    models)  (In response to:  Executive Order (EO) 13861 (the
%    President’s Roadmap to Empower Veterans and End a National Tragedy
%    of Suicide or PREVENTS), On signed on March 5, 2019)
%\end{itemize*}
