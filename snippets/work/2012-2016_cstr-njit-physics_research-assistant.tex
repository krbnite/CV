\leftandright{NJIT Center for Solar-Terrestrial Research}{Newark, NJ \textbullet\, January
2012 -- May 2016} \\  
\vspace{-0.8em}
\textit{Research Scientist} 
\begin{itemize*}
  \item Fusion and harmonization of multi-modal datasets, including 
    data deriving from: ground-based, globally-distributed instrument
    networks; high-altitude satellites, such as NOAA's DMSP fleet
    and NASA's Van Allen Probes; and interplanetary spacecraft, such as
    NASA's Advanced Composition Expolorer (ACE)
  \item Development of a data processing and modeling pipeline
     driven by solar wind inputs recorded upstream (sunward) of
     Earth by ACE which is able to forecast geomagnetic activity and
     solar wind energy deposits observable in Earth's deep polar cap region   
     with an hour lead time
  \item Proposal and design of an innovative remote-sensing technique
    that essentially inverts the modeling pipeline described above,
    instead taking the geomagnetic data sensed by a ground-based
    magnetometer located in the deep polar cap to drive
    a predictive model capable of continuously nowcasting state
    parameters of the concurrent solar wind flowing adjacent to the
    magnetosphere in near-Earth space.  Considering the strength and 
    persistence of the observed relationship, if necessary, I argue that
    a dedicated deep polar cap observatory can alleviate the need for
    new fleets of in-situ spacecraft 
  \item Mapping of and insight into the complex structure
    of the geomagnetic field in Earth's polar regions suggests 
    a revisitation of underappreciated geomagnetic coordinate systems
    and often ignored prospect of subterranean ground currents
    in the polar cap structuring and contributing to the polar
    cap hydromagnetic energy distributions
  %\item Implemented robust, non-parametric methods to characterize empirical distributions using as
  %  few assumptions as possible
  \item Development of unique visualization and analysis techniques
    to aid in studying the hydromagnetic structure of the high-latitude polar
    cap and depict its evolution in response to variations in near-Earth 
    space weather conditions.
  \item  Miscellaneous, ad hoc data analysis efforts as a member
    of the RBSPICE instrument team for NASA's Van Allen
    Probes mission  
  \item Software packages developed in Bash, R, MatLab, and IDL to enable efficient, 
    streamlined analyses and reporting for a range of ongoing projects 
  \item First-author publication following defense of PhD dissertation
% \item Through the combination of data/observationally-driven research and deep
%   physical understanding of magnetohydrodynamics, explained
%   decades-old problem that prevented proper understanding of the geomagnetic field's
%   complex structure in Earth's polar regions 
%   \textsuperscript{\href{http://www.slideshare.net/inverseuniverse/polarcappower}
%   {[link]}}. The research involved stochastic analysis; development of
%   novel visualization and analysis techniques
%   \textsuperscript{\href{http://www.slideshare.net/inverseuniverse/spectralproducts-64254698}{[link]}};
%   %IDL__8day_multi-mag-spectra.pdf}{example}};
%   collection, harmonization, and normalization of 100's of data sets 
%   \textsuperscript{\href{http://kevin-urban.com/video/R__20130302_trifecta.mp4}{[link]}}
%   distributed across the Earth and solar system to model and
%   understand the linear/nonlinear drivers of complex, physical
%   systems.
% \item The product development included statistical and physical model
%   selection, new interpretations of results commonly understood as
%   anomalies found in the empirical spatial distribution, hypothesis
%   testing, and development of dynamic correlation metrics which verified 
%   time-varying causal relationships 
%   \href{http://kevin-urban.com/images/dynamic-correlograms.pdf}
%   {\textsuperscript{[link]}}, thus verifying the 
%   contrary hypothesis to previous understanding of the subject matter.
\end{itemize*}

