
\leftandright{Early Signal (Cohen Veterans Bioscience)}{New York, NY
\textbullet\, July 2018 -- Present} \\  

\leftandright{\emph{Director of Data Science \& Digital Health
(July 2021 - Present)}}{} \\

\leftandright{\emph{Associate Director of Data Science \& Digital Health
(June 2019 - July 2021)}}{} \\

\leftandright{\emph{Senior Data Scientist (July 2018 - Jun 2019)}}{} \\

Research and development efforts at the intersection of sensors, deep
learning, and brain health. 
This includes the design and testing of signal processing and machine learning 
pipelines on out-of-lab, ''real world'' sensor datasets, including devies such as 
wearables, mobile phones, and smart home tech, as well as in-lab medical
devices. The devices collect raw sensor data from 
one or more sensors (e.g., accelerometer, gyroscope, magnetometer, PPG,
heart rate, EDA/GSR, EEG, temperature). The overarching theme across multiple
brain-health domains I work on (e.g., Parkinson's disease, Rett syndrome, PTSD, TBI, 
suicide) is to detect, classify, track, and/or predict semantically-meaningful 
features related to physical activity, sleep quality, symptoms, disease 
progression, and brain/mental health (e.g., human activity recognition, gesture recognition, 
symptom monitoring, clinical scale estimation/forecasting), and to do
this in an objective, consistent, high-frequency manner. These projects
often include more traditional measures and datasets as well, such as
clinical scales, neurocognitive tests, health records, demographics, 
brain imaging, etc.

\vspace{0.2in}
Other activities include project management, help with grant writing,
mentoring, landscaping and literature reviews, poster presentations, 
and publishing (several papers in early stages of peer review 
process).

\vspace{0.2in}
Sample of Externally-Facing Work
\begin{itemize*}
   \item Are wearables worth the hype? 
     \url{https://www.cohenveteransbioscience.org/2020/09/29/are-wearables-worth-the-hype-digital-biomarkers/}
   \item  Quantifying stereotypic hand movements in Rett Disorder. 
     Society of Biological Psychiatry's 74th Annual Scientific Conference
     (Innovations in Clinical Neuroscience: Tools, Techniques and
     Transformative Frameworks). 2019.
   \item  Quantifying stereotypic hand Movements in rare disorders using
     IMU sensors and deep learning algortihms.
     National Organization for Rare
     Disorders (NORD): Rare Diseases and Orphan Products
     Breakthrough Summit. 2019.
   \item  Suicide report submitted to the White House in response to a Request 
     for Information (RFI): Executive Order (EO) 13861  - President’s Roadmap 
     to Empower Veterans and End a National Tragedy of Suicide (PREVENTS)
\end{itemize*}

