\leftandright{NJIT Center for Solar-Terrestrial Research}{Newark, NJ \textbullet\, January
2012 -- May 2016} \\  
\vspace{-0.8em}
\textit{Research Scientist} 
\begin{itemize*}
  \item Fusion and harmonization of data sets from a ground-based,
    globally-distributed network of instruments; high-altitude 
    satellites, such as NOAA's DMSP fleet and NASA's Van Allen
    Probes; and interplanetary spacecraft (NASA's ACE)
  \item Provided new, empirical insights and interpretation of
    decades-old problem concerning the geomagnetic field's complex
    structure in Earth's polar regions
  \item Developed prediction scheme to forecast intensity and
    spatial-temporal distribution the magnetic weather
    in Earth's deep polar cap region using spacecraft upstream of Earth in the
    solar wind (NASA's ACE) 
  \item Inverted the prediction scheme to leverage it as a remote-sensing technique 
    such that data from ground-based magnetometers in Antarctica can
    be used to infer space weather parameters in near-Earth space and,
    if necessary, could essentially replace in-situ spacecraft for certain measurements
  %\item Implemented robust, non-parametric methods to characterize empirical distributions using as
  %  few assumptions as possible
  \item Created innovative visualization and analysis techniques
    to better understand and transform our understanding of the 
    Earth magnetic geography and its dynamic response to varying
    space weather conditions
  \item  Supported myriad data analysis efforts and projects as a member
    of the RBSPICE instrument team for NASA's Van Allen
    Probes mission, which has been essential in understanding and
    forecasting hazardous conditions of Earth's radiation belt
    environment
  \item Developed software packages in Bash, R, MatLab, and IDL to enable efficient, 
    streamlined analyses and reporting for a range of ongoing projects 
  \item Published first-author publication, PhD dissertation, and
    currently have several papers in development
% \item Through the combination of data/observationally-driven research and deep
%   physical understanding of magnetohydrodynamics, explained
%   decades-old problem that prevented proper understanding of the geomagnetic field's
%   complex structure in Earth's polar regions 
%   \textsuperscript{\href{http://www.slideshare.net/inverseuniverse/polarcappower}
%   {[link]}}. The research involved stochastic analysis; development of
%   novel visualization and analysis techniques
%   \textsuperscript{\href{http://www.slideshare.net/inverseuniverse/spectralproducts-64254698}{[link]}};
%   %IDL__8day_multi-mag-spectra.pdf}{example}};
%   collection, harmonization, and normalization of 100's of data sets 
%   \textsuperscript{\href{http://kevin-urban.com/video/R__20130302_trifecta.mp4}{[link]}}
%   distributed across the Earth and solar system to model and
%   understand the linear/nonlinear drivers of complex, physical
%   systems.
% \item The product development included statistical and physical model
%   selection, new interpretations of results commonly understood as
%   anomalies found in the empirical spatial distribution, hypothesis
%   testing, and development of dynamic correlation metrics which verified 
%   time-varying causal relationships 
%   \href{http://kevin-urban.com/images/dynamic-correlograms.pdf}
%   {\textsuperscript{[link]}}, thus verifying the 
%   contrary hypothesis to previous understanding of the subject matter.
\end{itemize*}

