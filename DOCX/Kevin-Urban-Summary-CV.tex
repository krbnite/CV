% vim:tw=72 sw=2 ft=tex
%         File: text.tex
% Date Created: 2015 Jul 27
%  Last Change: 2016 Jul 21
%     Compiler: pdflatex
%       Author: Kurban

\documentclass[10pt]{article}


%%%%%%%%%%%%%%
%% Packages %%
%%%%%%%%%%%%%%

%% For changing the margin size. %%
\usepackage[top=0.5in,bottom=0.5in,left=0.5in,right=0.5in]{geometry}

%% For colored text. %%
% -  See
%    http://en.wikibooks.org/wiki/LaTeX/Colors#The_68_standard_colors_known_to_dvips
%    for a list of predefined colors.
% \textcolor{colYouWant}{text}
\usepackage[usenames,dvipsnames]{color}

%% For providing a link to your email. %%
\usepackage[colorlinks=true,urlcolor=blue]{hyperref}

%% For making compact lists.
% -  use itemize* environment for compact itemized lists, etc.
\usepackage{mdwlist}



%%%%%%%%%%%%%%%%%%%%%%%%%%%%%%%%%%
%%             New              %%
%%           Commands           %%
%%%%%%%%%%%%%%%%%%%%%%%%%%%%%%%%%%

%% For making a new section in your resume. %%
% -  \noindent prevents LaTeX from indenting.
% -  \large makes the text slightly larger.
%    See http://en.wikibooks.org/wiki/LaTeX/Fonts#Sizing_text
%    for a list of other font-sizing commands.
% -  \vspace adds a little bit of vertical space.
% -  \hrule adds a horizontal line.
\newcommand{\ressection}[1]{\noindent{\large\textbf{#1}}
\vspace{2pt}\hrule\vspace{4pt}}

%% For job titles and date names. %%
% -  \hfill fills up horizontal space until
%    the words to the left and the right are
%    as far apart as possible.
\newcommand{\leftandright}[2]{\noindent\textbf{#1}\hfill
\textbf{#2}}



%%%%%%%%%%%%%%%%%%%
%% Miscellaneous %%
%%%%%%%%%%%%%%%%%%%

\pagestyle{empty} %% removes the page numbers.



%%%%%%%%%%%%%%%%%
%% The Resume! %%
%%%%%%%%%%%%%%%%%


\begin{document}

\leftandright{\textbf{\huge{Dr. Kevin Urban}}}
{\url{http://www.kevin-urban.com}}

\leftandright{kevin.ddu@gmail.com \textbullet\, (973) 464-6833}
{\url{https://www.linkedin.com/in/drkurban}}


%========================================================
%     PROFILE / SUMMARY
%========================================================
\vspace{0.2cm}
\ressection{Profile}
\noindent Innovative data scientist and physicist with proven experience in 
data product development and integrated, cohesive visualization of complex,
multi-source, high-dimensional data sets. Intuition for trends and
patterns which yield discerning insights and hypotheses, while
strategic insight and methodology result in rapid research-to-product
development. Collaborative personality with the understanding that 
a progressive, open-minded work environment and
coupled skill sets result in efficient, high-value, dependable data
products. Effective written and verbal communicator.


%========================================================
%    DATA / PROGRAMMING LANGUAGES
%========================================================
\vspace{0.3cm}
\ressection{Computing Skills}
\noindent\textbf{Programming / Tools}: R, MatLab, Python, Interactive
  Data Language [IDL], UNIX, Bash (Shell Scripting),   
  JavaScript, SQL, LaTeX, Vim, Fortran, Spark, Hive, MongoDB, Java, C/C++
  \vspace{0.1in}

\noindent\textbf{Analytics}:
\vspace{-0.2cm}
\begin{itemize*}
  \item \textbf{Structured Data} (e.g., uni/multi-variate time series,
    geospatial, model/simulation output, genomics data)
  \item \textbf{Unstructured Data} (e.g., image analysis, text mining / scrubbing
    web pages)
  \item \textbf{Sensor/Instrument Data Sets} 
    (e.g., mining and analysis of vectorial time series data
    from 100's of globally-distributed geomagnetic sensors)
  \item \textbf{Data Pre-Processing} (detection and
    removal/remedy of outliers, clipped or out-of-range values, missing
    values; data structuring, normalization, and/or transformations)
  \item \textbf{Probability \& Statistics} (e.g., sample statistics, inference, model
    identification, hypothesis testing, resampling, bootstrapping, confidence
    intervals, conditioning, error analysis, statistical modeling)
  \item \textbf{Supervised Machine Learning} (e.g., regression, classification, clustering,
    prediction)
  \item \textbf{Unsupervised Machine Learning} (e.g., PCA, SVD,
    hierarchical clustering, dimensionality reduction)
  \item \textbf{Time Series Analysis} (spectral
    analysis, digital signal processing, ARIMA, GARCH, prediction / forecasting,
    non-parametric methods)
  \item \textbf{Anomaly Detection} (e.g., kernel density methods,	
    event detection, pattern recognition)
  \item \textbf{Statistical / Numerical Modeling}
    (model selection, linear/nonlinear models, dynamical systems,
    differential/difference equations, numerical stability, eigenvalues,
    Runge-Kutta methods)
\end{itemize*}






%========================================================
%  WORK / RESEARCH   EXPERIENCE
%========================================================
\vspace{0.2cm}
\ressection{Work and Professional Research Experience}

\leftandright{Department of Physics, NJIT}{Newark, NJ \textbullet\, January
2012 -- present} \\  
\vspace{-0.8em}
\textit{Research Scientist, Ph.D. Candidate. Advisors: Andrew Gerrard,
Louis Lanzerotti, Denis Blackmore} 
\begin{itemize*}
  \item 
    Through the combination of data/observationally-driven research and deep
    physical understanding of magnetohydrodynamics, explained
    decades-old problem which prevented proper understanding of the geomagnetic field's
    complex structure in the polar regions. The research involved
    stochastic analysis; development of novel visualization and
    analysis techniques; collection, harmonization, and normalization of
    100's of data sets distributed across the Earth and solar system
    to model and understand the linear/nonlinear drivers of complex,
    physical systems.
  \item The product development included statistical and physical model
    selection, new interpretations of results commonly understood as
    anomalies found in the empirical spatial distribution, hypothesis
    testing, and development of dynamic correlation metrics which verified 
    time-varying causal relationships, thus verifying the 
    contrary hypothesis to previous understanding of the subject matter.
\end{itemize*}


%-------------------------------------------------------------
\leftandright{Department of Mathematical Sciences, NJIT}{Newark, NJ \textbullet\, 2008 -- 2012}\\  
\vspace{-0.8em}
\textit{Research Scientist, Professor Denis Blackmore} 
\begin{itemize*}
  \item
    Investigated sources of chaos in a 
    dimensionally-reduced parameter space of a discrete-dynamical system via
    \href{http://kevin-urban.com/video/R__gamma-var_restitution-0.8.gif}{Poincare
    map simulations in R} 
  \item
    Numerically modeled and contrasted several representations of granular fluid systems
    \href{http://www.kevin-urban.com/images/gran_tapping-pics.png}{in
    MatLab}. 
  \item Worked with numerically stiff difference equations requiring a stability
    analysis to understand how to minimize the propagation of numerical errors, which otherwise
    prevent proper quantitative analysis and visualization of the underlying dynamical system.
  \item
    \href{http://msp.org/jomms/2011/6-1/jomms-v6-n1-p06-s.pdf}{Published
    papers} on
    \href{http://www.sciencedirect.com/science/article/pii/S0167278914000189}{granular
    fluid systems} and 
    \href{http://www.icmp.lviv.ua/journal/zbirnyk.64/}{fractional
    dynamics} 
  \item
    Gained knowledge in advanced mathematical disciplines such as
    dynamical systems, topological analysis, manifold theory, abstract
    algebra, stochastic calculus
  %\item Gained practical experience using the R programming language
\end{itemize*}

% \vspace{0.25cm}
% \leftandright{Department of Physics and Astronomy, Siena
% College}{Loudonville, NY \textbullet\, August 2011 -- January 2012}\\  
% \vspace{-0.8em}
% \textit{Research Scientist, Contractor} 
% \begin{itemize*}
%   \item
%     Continued development of
%     \href{http://kevin-urban.com/images/IDL__Pretty-PSD.pdf}{time series, spectral, 
%     and regression data analysis techniques} in effort to delineate open
%     and closed geomagnetic flux using magnetometers situated in and
%     around the southern polar cap
% \end{itemize*}


%-------------------------------------------------------------
\leftandright{NASA/CalTech Jet Propulsion Laboratory}{Pasadena, CA \textbullet\, Summer 2011}\\  
\vspace{-0.8em}
\textit{Intern, Trajectory Optimization / Design} 
\begin{itemize*}
  \item
    Participated in the development of a full-fledged spacecraft mission to
    the Trojan asteroids of Jupiter
    (from % responding to an Announcement of Opportunity [AO]
    establishing and prioritizing science goals, to optimizing the science-engineering-financial
    parameter space, to the written proposal and presenting our mission design to the NASA review
    board)
  \item 
    Gained an appreciation of rapid product development via concurrent engineering
  \item Gained understanding of the intricate interplay between the
    various engineering system designs, science goals, timeline
    requirements, and budget constraints.
  %\item
  %  Gained understanding of the interplay between the trajectory design
  %  and other considerations, such as launch
  %  windows, budget time lines, a multi-component propulsion system,
  %  and payload
  %\item 
  %  Helped develop the science time line and traceability matrix for the
  %  baseline (desired) and threshold (minimum requirement) missions

\end{itemize*}
   
%-------------------------------------------------------------
\leftandright{Department of Physics, NJIT}{Newark, NJ \textbullet\, 
August 2008 - August 2010}\\  
\vspace{-0.8em}
\textit{Research Scientist, M.S. Student. Advisors: Andrew Gerrard,
Louis Lanzerotti} 
\begin{itemize*}
  \item
    Analyzed multi-channel instrument data from a spatially-distributed
    network of automated observatories in Antarctica.
  \item Developed data metrics and associated classification scheme for
    events of interest in the data sets.
  \item Demonstrated the feasibility of a real-time
    detection/classification scheme in maintaining surveillance on
    local, regional, and global-scale geomagnetic events of interest.
  \item Gained practical experience in various data analysis techniques,
    such as digital signal processing, time series analysis, spectral analysis,
    statistical methods, and regression.
  \item In-depth experience visualizing and analyzing data in R, MatLab,
    and IDL
  \item Published Master's thesis
    %: Synoptic variability of a CIR-driven
    %open-closed boundary during solar minimum.
\end{itemize*}

%-------------------------------------------------------------
\leftandright{NASA Goddard Space Flight Center }{ }
\leftandright{\textcolor{white}{space} }{Greenbelt, MD \textbullet\, June -- August 2007}\\  
\vspace{-0.8em}
\textit{Intern, Observational Cosmology Laboratory} 
\begin{itemize*}
  \item Developed software implementing a Stokes parameter analysis to
    run on simulation data for the proposed Absolute Spectrum
    Polarimeter [ASP] — an instrument designed to detect B-mode
    gravitational waves, which would provide evidence for Einstein’s
    theory of general relativity and cosmological inflationary theory
  \item Gained practical experience in IDL, MatLab, and shell scripting
  % \item Studied cosmology as it related to the mission goals of ASP
\end{itemize*}


%-------------------------------------------------------------
\leftandright{NASA Goddard Space Flight Center}{Greenbelt, MD \textbullet\, June -- August 2006}\\  
\vspace{-0.8em}
\textit{Intern,  Heliophysics Division}
\begin{itemize*}
  \item Worked extensively in the UNIX programming environment, learning and
    developing Python, HTML, CSS, JavaScript, and PHP code
  \item Developed web content  % for the International Heliophysical Year [IHY]
\end{itemize*}



%========================================================
%     PEER-REVIEWED PAPERS
%========================================================
\vspace{0.2cm}
\ressection{Selected Peer-Reviewed Publications}
\vspace{-0.8em}
\begin{itemize*}

  \item
    % Urban, K. D., A. J. Gerrard, L. J. Lanzerotti, and A. T. Weatherwax, 
    Penetration of solar wind hydromagnetic
    energy deep into the polar cap: remote detection and forecasting. 
    Geophysical Research Letters,
    Submitted, 2016.

  \item
    % Urban, K. D., A. J. Gerrard, L. J. Lanzerotti, and A. T. Weatherwax, 
    Rethinking the polar cap: eccentric-dipole structuring of ULF power
    at the highest corrected geomagnetic latitudes. Journal of Geophysical
    Research, In Press, 2016.

  %\item \label{itm:gran1}
  %  %Blackmore, D., A. Rosato, X. Tricoche, K. D. Urban, and L. Zou,
  %  \href{http://www.sciencedirect.com/science/article/pii/S0167278914000189}{ Analysis, 
  %  simulation, and visualization of 1D tapping via
  %  reduced dynamical models}. Physica D: Nonlinear Phenomena, Vol.
  %  273, 2014.

  \item
    %Diniega, S., K. Sayanagi, J. Balcerski, B. Carande, R. Diaz-Silva,
    %A. Fraeman, S. Guzewich, J. Hudson, A. Nahm, S. Potter, M. Route, K.
    %D. Urban, S. Vasisht, B. Benneke, S. Gil, R. Livi, B. Williams, and C.
    %Budney, L. Lowes,   
    \href{http://www.sciencedirect.com/science/article/pii/S0032063312003741}{Mission 
    to the Trojan Asteroids: lessons learned
    during a JPL Planetary Science Summer School mission design
    exercise}, Planetary and Space Science, Vol. 76, 2013.

%  \item 
%    %Rosato, A., D. Blackmore, X. M. Tricoche, K. D. Urban, and L. Zuo,
%    \href{http://scitation.aip.org/content/aip/proceeding/aipcp/10.1063/1.4811931}{Dynamical
%    systems model and discrete element simulations of a tapped granular
%    column}. Powders and Grains 2013: Proceedings of
%    the 7th International Conference on Micromechanics of Granular
%    Media. Vol. 1542. No. 1. AIP Publishing, 2013.

  \item
    %Urban, K. D., Y. Bhattacharya, A. J. Gerrard, A. Ridley, L. Lanzerotti,
    %and A. Weatherwax, 
    \href{http://www.agu.org/pubs/crossref/2011/2011SW000688.shtml}{Quiet-time
    observations of the open-closed boundary
    prior to the CIR-induced storm of August 9, 2008}. Space Weather,
    Vol. 9, S11001, 2011.

%  \item \label{itm:gran2}
%    %Blackmore, D., A. Rosato, X. Tricoche, K. Urban, and V. Ratnaswamy,
%    \href{http://msp.berkeley.edu/jomms/2011/6-1/jomms-v6-n1-p06-s.pdf}{Tapping
%    dynamics for a column of particles and beyond}. Journal of
%    Mechanics of Materials and Structures, Vol. 6, No. 1-4, 2011.

%  \item
%    % Urban, Kevin, 
%    ``Synoptic Variability of a CIR-driven Open-closed
%    Boundary During Solar Minimum.'' Master's Thesis. New Jersey Institute of
%    Technology, Department of Physics, 2010.

%  \item \label{itm:gran3}
%    % Blackmore, D., K. Urban, and A. Rosato,
%    \href{http://www.icmp.lviv.ua/journal/zbirnyk.64/}{Integrability analysis of regular and
%    fractional Blackmore-Samulyak-Rosato fields}.  Condensed Matter
%    Physics, Vol. 13, No. 4, 2010. 
\end{itemize*}



%========================================================
%     EDUCATION
%========================================================
\vspace{0.2cm}
\ressection{Education}

\leftandright{New Jersey Institute of Technology / Rutgers University,
\textmd{Ph.D. Physics}}{Newark, NJ
\textbullet\, May 2016} \\  
\leftandright{}{Dissertation: \textmd{The Hydromagnetic Structure of the
Polar Cap and Its Interaction with the Solar Wind}}\\  

\leftandright{New Jersey Institute of Technology / Rutgers University,
\textmd{M.S. Physics}}{Newark, NJ
\textbullet\, August 2010} \\  
\leftandright{}{Thesis: \textmd{Synoptic Variability of a CIR-Driven Open-Closed
Boundary During Solar Minimum}}\\  

\leftandright{New Jersey Institute of Technology / Rutgers University,
\textmd{B.S. Physics}}{Newark, NJ
\textbullet\, May 2008} \\  


%========================================================
%    COMMUNICATION
%========================================================
%\ressection{Miscellaneous}
%\noindent Effective written and verbal communicator, e.g.,
%experiences include:
%\vspace{-0.3cm}
%\begin{itemize*}
%  \item published author
%  \item news writer 
%  \item physics and mathematics tutor
%  \item guitar instructor 
%  \item performing musician
%  \item scored in the 99th percentile in the verbal section of the
%    Graduate Record Examination (GRE)
%\end{itemize*}

%========================================================
%    COURSEWORK
%========================================================
%\ressection{Relevant Coursework}
%\noindent\textbf{Graduate}:
%    Real Analysis, Complex Analysis, Topology, Differentiable Manifolds,
%    Applications of Abstract Algebra, Stochastic Calculus, 
%    Radio Astronomy, Stellar Magnetism, Physics of the Magnetosphere/Ionosphere System,
%    Atmospheric Physics, Electrodynamics, Statistical Mechanics,
%    Classical Mechanics, Quantum Mechanics, Quantum Electrodynamics
%\par\vspace{0.5em}
%\noindent\textbf{Undergraduate}:
%    Data Reduction, Probability/Statistics, Linear Algebra, Computer
%    Science I-II (Java/C++), Abstract Algebra, Vector Calculus,
%    Differential Equations, Advanced Calculus, Electromagnetism,
%    Thermodynamics, Classical Mechanics, Observational Astronomy,
%    Astronomy, Astrophysics I-II,  Quantum Mechanics, Special
%    Relativity, General Relativity
%\par\vspace{0.5em}
%\noindent\textbf{Extracurricular}
%    Computing for Data Analysis (R) [Coursera, R. Peng], Data Analysis
%    (R) [Coursera, J. Leek], Bioinformatics Algorithms [Coursera, P.
%    Pevzner], Machine Learning [Coursera, A. Ng], Operator Algebras and
%    Conformal Field Theory [Summer School, U. Oregon], Concurrent
%    Engineering [Planetary Science Summer School, Jet Propulsion
%    Laboratory]



\end{document}




